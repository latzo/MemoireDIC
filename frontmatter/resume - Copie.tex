
\chapter*{Résumé}
Afin d’appliquer les notions enseignées en Cours de Systèmes Multi Agents, nous avons eu à réaliser une étude de cas avec GAMA,un environnement de développement orienté modélisation et simulation de systèmes à base d'agents. Cet étude de cas concerne une \textbf{\textit{exploitation d'un territoire minier par une population de robots}}. Les robots doivent trouver tous les minerais présents dans le territoire et les amener à une base.\\
L'étude de cas est décliné en deux versions : une première version avec des robots réactifs et une deuxième version avec des robots cognitifs. Nous avons structuré notre rapport de la façon suivante :
Tout d'abord, dans une première partie nous faisons une spécification générale dans laquelle nous fixons les définitions de chacun des éléments manipulés dans notre étude de cas. Puis dans une seconde partie, nous présentons la première version du robot, le robot réactif. Ensuite, dans une troisième partie, nous déclinerons le robot cognitif. Puis nous ferons des test de sensibilité et comparaisons des deux versions avant de finir par une conclusion. \\
Balancing of an inverted pendulum is a classical problem in the field of Control Theory and Engineering. It's balancing is always challenging for the beginners in control engineering. This thesis deals with stabilization and control of both the linear and rotary inverted pendulum  systems. Simultaneous approach is applied to system analysis as well as controller synthesis. Great similarity of both the systems is pointed out during the derivation of the equations of motion using the Euler-Lagrange equation. Linear inverted pendulum system is analyzed in two experimental setups , pendulum on cart and two wheel self balancing vehicle. PID controller is used to control the system by trial and error based tuning. Potentiometer (pot) and gyroscopic sensor is used as feedback sensor of the controlled system. Rotary inverted pendulum system is analyzed in a experimental setup developed by $QUANSER$ and is balanced by Pole placement method. Rotary encoder is used as the feedback sensor of the controlled system. Later, Linear Quadratic Regulator (LQR) is designed for optimum control of the pendulum. System response and controller gains are simulated in $MATLAB$ environment and after applying controller in experimental prototype, actual response of the system is reported. \\ 
\\ \textbf{Keywords}: Inverted pendulum,  PID controller, Pole placement method, LQR.
\clearpage
