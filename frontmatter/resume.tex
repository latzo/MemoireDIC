
\chapter*{Résumé}

La mise en place d'applications Web et Mobile sécurisées est un problème classique pour beaucoup d'entreprises. L’augmentation rapide de la connectivité, combinée à l’augmentation spectaculaire de la valeur des données manipulées par ces applications ainsi que l’utilisation croissante de nouveaux protocoles et technologies ont abouti à des applications exposant à la fois les organisations qui les mettent en place ainsi que leurs utilisateurs à des risques considérables.\\
Bien que le terme « sécurisé » soit assez relatif, il n'en demeure pas moins qu'un certain degré de sécurité est indispensable pour ces applications. Pour y arriver, une approche idéale est de considérer la sécurité comme partie intégrante du projet pendant tout son cycle de vie. Encore, faudrait-il que les parties tenantes en soient conscientes et que les ressources permettant d'atteindre cet objectif soient disponibles.\\
Ce document est un mémoire de fin de cycle en vue de l’obtention du Diplôme d’Ingénieur de Conception en Informatique. Il présente la réalisation d’un projet effectué dans le cadre d’un stage à la société HubSo. Ce projet consiste en l'étude et la mise en œuvre d'une application Web Java EE et d'une application Mobile Android conformes OWASP.\\
Nous avons conçu une bibliothèque de fonctions de sécurité intégrable aux applications web et mobiles. Notre solution est basée sur une API préexistante opensource développée par OWASP, OWASP ESAPI. En outre une intégration de la bibliothèque à une application existante a été réalisée.\\
La bibliothèque a été développée en Java SE et un guide développeur a été proposé de même qu'une architecture d'application conforme OWASP.\\ \\
\textbf{Mots clés}: Sécurité, OWASP, Java



\clearpage
