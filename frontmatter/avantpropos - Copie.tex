%
% File: declaration.tex
% Author: V?ctor Bre?a-Medina
% Description: Contains the declaration page
%
% UoB guidelines:
%
% Author's declaration
%
% I declare that the work in this dissertation was carried out in accordance
% with the requirements of the University's Regulations and Code of Practice
% for Research Degree Programmes and that it has not been submitted for any
% other academic award. Except where indicated by specific reference in the
% text, the work is the candidate's own work. Work done in collaboration with,
% or with the assistance of, others, is indicated as such. Any views expressed
% in the dissertation are those of the author.
%
% SIGNED: .............................................................
% DATE:..........................
%
\chapter*{Avant-propos}
\begin{SingleSpace}

L’Université Cheikh Anta Diop de Dakar (UCAD) regroupe en plus de ses facultés des instituts
et écoles dont l’École Supérieure Polytechnique de Dakar (ESP). L’ESP a été créée le 24 Novembre
1994 suite aux recommandations de la concertation nationale sur l’enseignement supérieur qui
s’est tenue d’Avril 1992 à Août 1993 et préconisant la reconstruction des écoles d’enseignements
technologiques de l’UCAD. Elle est organisée en six départements : Génie Chimique et Biologie
Appliquée, Génie Civil, Génie Électrique, Génie Informatique, Génie Mécanique et Gestion.
Le département Génie Informatique propose un enseignement fondamental (connaissances
des concepts de base et des méthodologies), évolutif (intégration de nouveaux concepts et des
besoins du marché) et ouvert (développement des capacités d’organisation et d’autonomie).
Dans le cadre de leur formation, et pour l’obtention du Diplôme d’Ingénieur de Conception
en informatique de l’École Supérieure Polytechnique de Dakar, les élèves-ingénieurs en dernière
année effectuent un stage. Ce stage devrait leur permettre :
\\— De consolider leur formation et d’avoir un environnement de pratique des notions étudiées
durant la formation.
\\— D’avoir un aperçu de la vie professionnelle.
\\— De travailler sur un projet de son élaboration jusqu’à sa finalisation.
\\Ce stage aboutit à la rédaction d’un mémoire suivi d’une soutenance devant un jury sanctionnant
ces années d’études et de formation.
\\Dans cette optique, nous avons effectué un stage au sein de Hubso ... . Le présent document
est le résultat de notre travail sur la ...

\end{SingleSpace}
%\clearpage