
\chapter*{Abstract}
%Balancing of an inverted pendulum is a classical problem in the field of Control Theory and Engineering. It's balancing is always challenging for the beginners in control engineering. This thesis deals with stabilization and control of both the linear and rotary inverted pendulum  systems. Simultaneous approach is applied to system analysis as well as controller synthesis. Great similarity of both the systems is pointed out during the derivation of the equations of motion using the Euler-Lagrange equation. Linear inverted pendulum system is analyzed in two experimental setups , pendulum on cart and two wheel self balancing vehicle. PID controller is used to control the system by trial and error based tuning. Potentiometer (pot) and gyroscopic sensor is used as feedback sensor of the controlled system. Rotary inverted pendulum system is analyzed in a experimental setup developed by $QUANSER$ and is balanced by Pole placement method. Rotary encoder is used as the feedback sensor of the controlled system. Later, Linear Quadratic Regulator (LQR) is designed for optimum control of the pendulum. System response and controller gains are simulated in $MATLAB$ environment and after applying controller in experimental prototype, actual response of the system is reported. \\ %
%\\ \textbf{Keywords}: Inverted pendulum,  PID controller, Pole placement method, LQR.%
\clearpage
