\chapter*{Conclusion}
Dans le cadre de notre stage de fin d’étude du cycle DIC à HubSo, nous nous sommes intéressés à l'étude et la mise en œuvre d'une application Web Java EE et d'une application Mobile Android conformes OWASP. Nous avions comme objectifs : 
\begin{itemize}
	\itemtirait la mise en place d'une bibliothèque de sécurité intégrable et disponible pour les projets web et mobile ;
	\itemtirait la mise en place d'un guide de bonnes pratiques OWASP pour développeurs ;
	\itemtirait une formation sur l'utilisation de la bibliothèque.
\end{itemize}
Comme objectifs supplémentaires, nous avions :
\begin{itemize}
	\itemtirait la proposition d'une architecture type Owasp (Client Android, Serveur Java EE) ;
	\itemtirait l'intégration d'une application existante ;
	\itemtirait la mise en place d'un projet type Web Java EE et d'un projet type Android conformes Owasp avec l'authentification et la gestion de sessions, la gestion des utilisateurs et mots de passe, la gestion des profils et la génération des composants graphiques.
\end{itemize}
Pour atteindre ces objectifs, nous avons commencé par faire un état de l’art pour recueillir les bonnes pratiques en matière de prise en charge de la sécurité dans les applications web et mobile. Puis nous avons défini une méthodologie de développement adaptée à notre sujet, Scrum. Enfin, nous avons mis en place une solution. \\
Nous avons pu atteindre les objectifs suivants :
\begin{itemize}
	\itemcheck la mise en place de la bibliothèque de sécurité : elle a été faite grâce à l'utilisation de la bibliothèque OWASP ESAPI ;
	\itemcheck un guide de bonnes pratiques est en cours de finition ;
	\itemcheck une sensibilisation des développeurs aux risques de sécurité des applications web et mobiles a été faite grâce à l'affichage d'un document explicitant le Top 10 OWASP 2017 dans les locaux de HubSo ;
	\itemcheck l'une architecture type Owasp (Client Android, Serveur Java EE) a été proposée ;
	\itemcheck l'intégration d'une application existante a été faite, TouchWeb, qui sera auditée dans les prochains jours par BSSI;
\end{itemize}
Toutefois, les perspectives suivantes sont à envisager :
\begin{itemize}
	\itemtirait une formation sur l'utilisation de la bibliothèque ;
	\itemtirait l'extension de la bibliothèque par rapport aux opérations courantes de HubSo ;
	\itemtirait la mise en place des projets types Java EE et Android conformes Owasp ;
	\itemtirait le développement d'autres fonctions dans la bibliothèque car, bien que les dix risques présents dans le Top 10 OWASP soient les plus fréquents, une multitude d'autres risques existe et sont aussi dangereux pour les applications web et mobile ;
	\itemtirait l'audit de TouchWeb par un cabinet d'audit spécialisé qui se fera dans les prochains jours.
\end{itemize}.
Ce stage a été une étape très importante dans notre insertion dans le monde professionnel. Il a été l’occasion pour nous de mettre en pratique nos connaissances théoriques acquises tout au cours de notre cycle et de nous préparer aux réalités du nouveau monde vers lequel nous nous acheminons.