%\addcontentsline{toc}{chapter}{Introduction}

\chapter*{Introduction}
\addstarredchapter{Introduction}

La problématique de la sécurité des applications web et mobiles est actuellement un réel défi qui se dresse devant toute entreprise rigoureuse. Dans un contexte où ces applications sont devenues indispensables et revêtent une valeur capitale, elles sont, de plus en plus, la cible d'individus malintentionnés : cybercriminels, terroristes, ... Dés lors, les entreprises sont tenues de faire de la sécurité, un «must» dans tous leurs projets.\\
Ainsi, dans le cadre de notre stage à HubSo, il nous a été confié l'étude et la mise en œuvre d'une application web Java EE et d'une application mobile Android conformes OWASP. Et pour ce faire, nous serons amenés à mettre en place une bibliothèque de fonctions de sécurité. Cette bibliothèque devra pouvoir être intégrée facilement dans les applications existantes pour apporter des réponses aux problèmes de sécurité.
Ce présent document présente alors le travail réalisé lors de ce stage.
Le document s’articule autour de cinq grands chapitres que sont :
\begin{itemize}
	\itemcheck « Chapitre 1 : Présentation générale » : dans ce chapitre nous présentons la structure d’accueil ainsi que le sujet du mémoire.
	\itemcheck « Chapitre 2 : État de l’art » : dans ce chapitre, nous faisons
	l’état de l’art de la sécurité informatique en général et de la sécurité des applications web et mobiles en particulier. Nous étudions l'historique de la sécurité informatique, son contexte actuel et enfin OWASP qui fait un travail remarquable en matière de bonnes pratiques par rapport à la prise en charge de la sécurité dans les applications web et mobile.
	\itemcheck « Chapitre 3 : Méthodologie » : dans ce chapitre, nous présentons notre
	démarche de développement et de gestion du projet.
	\itemcheck « Chapitre 4 : Analyse et Conception » : dans ce chapitre, nous faisons dans un premier temps les spécifications pour ressortir les besoins fonctionnels et non fonctionnels de notre futur système, puis dans un deuxième temps, une analyse des besoins afin de faire ressortir la meilleure solution à apporter par rapport à la prise en compte au moins des risques énoncés dans le Top 10 OWASP  et enfin, la conception dans laquelle nous faisons des choix conceptuels, architecturaux et techniques permettant d’aboutir à une solution qui puisse satisfaire les besoins fonctionnels et non fonctionnels.
	\itemcheck « Chapitre 5 : Réalisation » : dans ce chapitre, nous présentons d’abord les outils et technologies que nous utiliserons pour implémenter notre solution, puis nous présentons les résultats qui en sont issus, et, en dernier lieu, nous présenterons l'intégration de notre solution dans une application existante.
\end{itemize}
Enfin, nous terminerons par la conclusion et les perspectives.
\clearpage
