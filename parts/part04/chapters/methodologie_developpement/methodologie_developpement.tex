\chapter{Méthodologie de développement}
\minitoc
\clearpage
\section{Qu'est ce qu'une méthodologie de développement ?}
A methodology formally defines the process that you use to gather requirements, analyze them, and design an application that meets them in every way. There are many methodologies, each differing in some way or ways from the others. \\
Une méthodologie est une démarche, une ligne de conduite qui est suivi par l’ensemble de l’équipe projet du lancement jusqu’à la livraison de celui-ci. Elle permet d’uniformiser le processus de travail et facilite la communication de l’équipe projet via des concepts et termes bien définis et compris par tous.
Une méthodologie est un processus, une démarche à suivre pour aboutir à la concrétisation d'un projet.
There are many reasons why one methodology may be better than another for your particular project: For example, some are better suited for large enterprise applications while others are built to design small embedded or safety-critical systems. On another axis, some methods better support large numbers of architects and designers working on the same project, while others work better when used by one person or a small group

\section{Intérêt d'une méthodologie}



                             
\section{Catégories de méthodologies de développement}
Les méthodes traditionnelles prônent un enchaînement séquentiel des différentes activités, depuis les spécifications jusqu’à la validation du système, selon un planning préétabli. Elles visent à mieux prédire la façon dont les choses « devraient » se passer. Malheureusement, cette vision rassurante est bien loin de la réalité des projets.\\
La conséquence est que plus de 80\% des projets exécutés selon ces méthodologies connaissent des retards, des dépassements budgétaires, quand ils ne finissent pas en échec total, pour n’avoir pas su satisfaire les attentes des clients [W11Agiles http://www.blog.erlem.fr/management/46-methodes-agiles-pourquoi-lesadopter].
Ces problèmes sont liés à plusieurs caractéristiques fondamentales de ces méthodologies :\\
- Le rôle joué par le client qui intervient principalement au moment du lancement du projet, à quelques jalons majeurs parfois espacés de plusieurs mois, et surtout en fin de projet pour la réception et la recette du système. Cet « effet tunnel » conduit souvent à une solution souvent inadaptée et de piètre qualité ;\\
- Le mode contractuel forfaitaire qui durcit les relations entre client et fournisseur, rend le passage de témoin long et douloureux à la fin du projet ; \\
- Une trop grande standardisation des activités d’ingénierie, dont l’enchaînement se révèle souvent inefficace. Formellement, les contrôles d'avancement et de qualité ne peuvent être menés que sur la base de documents dans les premières étapes, et bien des organisations sont devenues des usines à produire de la documentation au lieu de produire de la valeur (fonctions logicielles) pour les clients et les utilisateurs ; \\
- Le passage de relai entre les phases successives dans lesquelles oeuvrent très souvent des équipes différentes, généralise une relation de type client-fournisseur et n’encourage ni l’empathie ni l’esprit d’équipe, bien au contraire. Chaque transition se traduit par une perte de temps, de savoir, d’informations ou de responsabilité.\\
À l’opposé des approches traditionnelles, Les méthodes agiles utilisent un principe de développement itératif qui consiste à découper le projet en plusieurs étapes qu’on appelle « itérations ». Ces itérations ne sont rien d’autre que des mini-projets définis avec le client en détaillant les différentes fonctionnalités qui seront développées en fonction de leur priorité. Au lieu de consacrer beaucoup de temps à la planification, en essayant de tout prévoir, il suffit de se fixer un objectif plus modeste, réalisable dans un délai relativement court, et de planifier la suite des choses en fonction des résultats observés.L’agilité peut également dans ce cas améliorer les résultats déjà obtenus et faciliter la
résolution de bon nombre des difficultés vécues. Elle va amener les personnes impliquées à : \\
- Mieux collaborer, prendre du recul sur l’application en priorisant les actions ; \\
- Donner plus de visibilité aux clients et utilisateurs ; \\
- Éliminer « l’effet tunnel »  en le remplaçant par des itérations courtes et maîtrisées.\\
Il existe de nombreuses méthodes Agiles. Parmi celles-ci, nous avons entre autres XP (Extreme Programming), Scrum, RAD (Rapid Application Development), PUMA, etc.

\section{Choix d'une méthodologie de développement}

\clearpage 