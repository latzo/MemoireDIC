\chapter{Méthodologie}
\minitoc
\clearpage
\section{Méthodologie de développement}
\subsection{Qu'est ce qu'une méthodologie de développement ?}
Une méthodologie est une démarche, une ligne de conduite qui est suivi par l’ensemble de l’équipe projet du lancement jusqu'à’à la livraison de celui-ci. Elle permet d’uniformiser le processus de travail et facilite la communication de l’équipe projet via des concepts et termes bien définis et compris par tous.\\
Une méthodologie est un processus, une démarche à suivre pour aboutir à la concrétisation d'un projet. Une méthodologie définit formellement le processus à respecter pour rassembler les exigences, les analyser et concevoir une application qui les respecte à tous les égards. Il existe de nombreuses méthodologies, chacune différant d'une manière ou d'une autre des autres. \\
Il existe de nombreuses raisons pouvant faire qu'une méthodologie soit meilleure qu'une autre par rapport à un projet particulier : par exemple, certaines sont mieux adaptées aux grandes applications d'entreprise, tandis que d'autres sont conçues pour concevoir de petits systèmes intégrés ou à sécurité critique. D'un autre point de vue, certaines méthodologies supportent mieux un grand nombre d'architectes et de concepteurs travaillant sur le même projet, tandis que d'autres fonctionnent mieux lorsqu'elles sont utilisées par une seule personne ou par un petit groupe.
\subsection{Intérêt d'une méthodologie}   
L'intérêt de l'utilisation d'une méthodologie de développement dans la conduite d'un projet informatique se justifie par plusieurs facteurs :
\begin{itemize}
	\itemtirait De nombreux échecs de projets informatiques dans le passé dûs à un manque d'organisation, ou un non satisfaction des besoins ;
	\itemtirait La révolution de l'industrie logicielle engendrée par les échecs informatiques et qui
	introduit de nouveaux facteurs de validation de la qualité logicielle : le génie logiciel ;
	\itemtirait Les nombreuses exigences liées au coût, aux délais et à la complexité des projets informatiques.
\end{itemize}
L'utilisation de méthodologies de développement adaptées permet ainsi l'élaboration de systèmes informatiques de manière fiable et viable tout en répondant à l'ensemble des exigences du client et du génie logiciel.                          
\subsection{Catégories de méthodologies de développement}
Il existe plusieurs méthodologies de développement informatique. L’on distingue deux principalement
deux catégories de méthodologies qui se différencient par rapport à leurs approches : l’approche traditionnelle et l’approche agile. Les deux approches se distinguent essentiellement dans la manière de décomposer le projet.\\
Les méthodes traditionnelles prônent un enchaînement séquentiel des différentes activités, depuis les spécifications jusqu'à la validation du système, selon un planning préétabli. Elles visent à mieux prédire la façon dont les choses « devraient » se passer. Malheureusement, cette vision rassurante est bien loin de la réalité des projets.\\
La conséquence est que plus de 80\% des projets exécutés selon ces méthodologies connaissent des retards, des dépassements budgétaires, quand ils ne finissent pas en échec total, pour n’avoir pas su satisfaire les attentes des clients.\\
Ces problèmes sont liés à plusieurs caractéristiques fondamentales de ces méthodologies :
\begin{itemize}
	\itemtirait Le rôle joué par le client qui intervient principalement au moment du lancement du projet, à quelques jalons majeurs parfois espacés de plusieurs mois, et surtout en fin de projet pour la réception et la recette du système. Cet « effet tunnel » conduit souvent à une solution souvent inadaptée et de piètre qualité ;
	\itemtirait Le mode contractuel forfaitaire qui durcit les relations entre client et fournisseur, rend le passage de témoin long et douloureux à la fin du projet ;
	\itemtirait Une trop grande standardisation des activités d’ingénierie, dont l’enchaînement se révèle souvent inefficace. Formellement, les contrôles d'avancement et de qualité ne peuvent être menés que sur la base de documents dans les premières étapes, et bien des organisations sont devenues des usines à produire de la documentation au lieu de produire de la valeur (fonctions logicielles) pour les clients et les utilisateurs ;
	\itemtirait Le passage de relai entre les phases successives dans lesquelles oeuvrent très souvent des équipes différentes, généralise une relation de type client-fournisseur et n’encourage ni l’empathie ni l’esprit d’équipe, bien au contraire. Chaque transition se traduit par une perte de temps, de savoir, d’informations ou de responsabilité.
\end{itemize}
À l’opposé des approches traditionnelles, Les méthodes agiles utilisent un principe de développement itératif qui consiste à découper le projet en plusieurs étapes qu’on appelle « itérations ». Ces itérations ne sont rien d’autre que des mini-projets définis avec le client en détaillant les différentes fonctionnalités qui seront développées en fonction de leur priorité. Au lieu de consacrer beaucoup de temps à la planification, en essayant de tout prévoir, il suffit de se fixer un objectif plus modeste, réalisable dans un délai relativement court, et de planifier la suite des choses en fonction des résultats observés.L’agilité peut également dans ce cas améliorer les résultats déjà obtenus et faciliter la résolution de bon nombre des difficultés vécues. Elle va amener les personnes impliquées à :
\begin{itemize}
	\itemtirait Mieux collaborer, prendre du recul sur l’application en priorisant les actions ;
	\itemtirait Donner plus de visibilité aux clients et utilisateurs ;
	\itemtirait Éliminer « l’effet tunnel »  en le remplaçant par des itérations courtes et maîtrisées.
\end{itemize}
Il existe de nombreuses méthodes Agiles. Parmi celles-ci, nous avons entre autres XP (Extreme Programming), Scrum, RAD (Rapid Application Development), PUMA, etc.
\subsubsection{Scrum}
Scrum est une méthode agile de gestion de projete. Elle a pour objectif d’améliorer la cohésion de
l’équipe et la rapidité du processus de développement. Le nom Scrum renvoie à une pratique généralement connue au rugby signifiant la « mêlée ».\\
Le cycle de vie d’un projet Scrum peut être découpé en trois parties :
\begin{itemize}
	\itemcheck La phase d’initiation ou démarrage : il s’agit d’une phase linéaire où l’on
	définit le périmètre fonctionnel du système et la liste des fonctionnalités
	(Backlog) agencées par ordre de priorité, d’effort, de complexité et de risque.
	C’est aussi à ce niveau que l’architecture est définie ;
	\itemcheck La phase de développement est un processus empirique : le projet est découpé
	en cycles itératifs d’une durée de deux semaines ou sprints. Chaque sprint
	regroupe une ou plusieurs fonctionnalités du Backlog. Tout au long de cette
	phase, le travail réalisé est mesuré et contrôlé et une amélioration constante du
	prototype est faite ;
	\itemcheck La phase de clôture est une phase linéaire de gestion de la livraison du produit
	final.
\end{itemize}
Scrum est la méthode Agile la plus utilisée de nos jours \cite{scrum}. En bref, elle définit des rôles (le Scrum Master, le Product Owner et l’équipe de développement), dicte la réitération de sprints de production à durée limitée à la fin desquels des incréments fonctionnels de logiciel sont livrés et met en place des artefacts (le carnet de produit, le carnet de sprint, les graphiques d’avancement) ainsi que des cérémonies (planification de sprint, mêlée quotidienne, revue et rétrospective).
Scrum implique l’auto-organisation des équipes et permet beaucoup plus de réactivité pour s’adapter aux besoins (parfois changeants) du client. Elle sous-entend aussi l’application de principes Agiles, soit la transparence, la simplicité et la collaboration.\\
La méthode Scrum soutient la livraison rapide et régulière de fonctionnalités à haute valeur ajoutée.
Du fait de sa facilité d'utilisation, de l'optimisation de l'efficacité de ceux qui l'utilisent grâce à des mêlées quotidiennes permettant de lever tout obstacle ainsi que de ses nombreux autres avantages (Focus sur la qualité, transparence, l'adaptabilité au changement), Scrum est bien adapté à notre projet et HubSo est un grand adepte de Scrum. Ainsi, nous utiliserons Scrum pour mener à bien notre projet.
%\subsection{Choix d'une méthodologie de développement}
\begin{comment}
Scrum is an agile method for project management developed by Ken Schwaber. Its goal is to dramatically improve productivity in teams previously paralyzed by heavier, process-laden methodologies.
Scrum is characterized by:
A living backlog of prioritized work to be done.
Completion of a largely fixed set of backlog items in a series of short iterations or sprints.
A brief daily meeting (called a scrum), at which progress is explained, upcoming work is described, and obstacles are raised.
A brief planning session in which the backlog items for the sprint will be defined.
A brief heartbeat retrospective, at which all team members reflect about the past sprint.
Scrum is facilitated by a scrum master, whose primary job is to remove impediments to the ability of the team to deliver the sprint goal. The scrum master is not the leader of the team (as they are self-organizing) but acts as a productivity buffer between the team and any destabilizing influences.
Scrum enables the creation of self-organizing teams by encouraging verbal communication across all team members and across all disciplines that are involved in the project. A key principle of scrum is its recognition that fundamentally empirical challenges cannot be addressed successfully in a traditional “process control” manner. As such, scrum adopts an empirical approach - accepting that the problem cannot be fully understood or defined, focusing instead on maximizing the team's ability to respond in an agile manner to emerging challenges.
\end{comment}
