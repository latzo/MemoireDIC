\chapter{UML}

\section{Intérêt d'une modélisation}
Un modèle est une représentation abstraite et simplifiée d'une entité du monde réel en vue de le décrire, de l'expliquer ou de le prévoir. Modéliser, c’est décrire de manière visuelle et graphique les besoins et les solutions fonctionnelles et techniques d'un projet.\\
Concrètement, un modèle permet de réduire la complexité d'un phénomène ou d'une entité en éliminant les détails qui n'influencent pas son comportement de manière significative. Il reflète ce que le concepteur croit important pour la compréhension et la prédiction du phénomène modélisé. Les limites du phénomène modélisé dépendant des objectifs du modèle.\\
Modéliser un système avant sa réalisation permet de mieux comprendre le fonctionnement du système. C’est également un bon moyen de maîtriser sa complexité et d’assurer sa cohérence. Un modèle est un langage commun, précis, qui est connu par tous les membres de l’équipe et il est donc, à ce titre, un vecteur privilégié pour communiquer. Cette communication est essentielle pour aboutir à une compréhension commune  et précise d'un système par ses différentes parties prenantes.

\section{Présentation d'UML}
UML est l’acronyme de « Unified Modeling Language » qu'on peut traduire par « langage de modélisation unifié ». Il s'agit d'un langage de modélisation graphique et textuel, un outil de modélisation constitué d’un ensemble de schémas, appelés diagrammes UML, qui donnent chacun une vision différente du projet à traiter. En effet, un document texte décrivant de façon précise un système contiendrait plusieurs pages. En général, peu de personnes ont envie de lire ce genre de document. De plus, un long texte de plusieurs pages est source d’interprétations et d’incompréhension. UML nous aide à faire cette description de façon graphique et devient alors un excellent moyen pour « visualiser » le futur système.\\
UML utilise l'approche objet qui a déjà fait ses preuves. Il permet de faire une abstraction des technologies objet en permettant d’exprimer et d’élaborer des modèles objet, indépendamment de tout langage de programmation. L'aspect formel de sa notation, limite les ambiguïtés et les incompréhensions.\\
Son indépendance par rapport aux langages de programmation, aux domaines d'application et
aux processus, en fait un langage universel. En effet, le processus de collecte et d'analyse des exigences d'une application et de leur intégration dans la conception d'un programme, est complexe et il existe actuellement nombre de  méthodologies qui définissent des procédures formelles spécifiant la démarche à suivre. Une des caractéristiques d'UML est qu'il est indépendant de toute méthodologie. Quelle que soit la méthodologie de développement utilisée dans un projet, on peut utiliser UML pour la modélisation du système. Il a été pensé pour servir de support à une analyse des concepts objet. C’est un langage formel, défini par un méta-modèle.\\
UML est aussi un support de communication performant, qui facilite la compréhension de systèmes  aussi complexes qu'ils soient.\\
UML est le résultat de la fusion de trois méthodes orientées objet Booch, OMT (Object Modeling Technique) et OOSE (Object Oriented Software Engineering) conçues respectivement par Grady Booch, James Rumbaugh et Ivar Jacobson. UML a démarré avec la version 0.8 intégrant les méthodes BOOCH 93 et O.M.T. Par la suite ce fut l'avènement de la version 0.9 ayant intégré la méthode OOSE. La version 1.0, proposé à l'O.M.G en 1996, fut finalement standardisée en 1997 sous la version 1.1 . Depuis, il y a eu plusieurs révisions du standard. Les dernières améliorations étant conséquentes, UML est passé à une nouvelle version : UML 2.0 (ou UML 2), abrégé souvent en U2. En 2005, l'Organisation internationale de normalisation (ISO) a également publié UML en tant que norme ISO approuvée.

\section{Diagrammes}
UML propose 14 diagrammes qui sont dépendants hiérarchiquement et se complètent, de façon à permettre la modélisation d'un projet tout au long de son cycle de vie. Un diagramme UML est une représentation graphique, qui s'intéresse à un aspect précis du modèle. C'est une perspective du modèle. Ces diagrammes sont répartis en 3 grands groupes : \\
\textgreater Diagrammes structurels ou statiques : \\
\textbullet Diagramme de classes : il représente les classes intervenant dans le système et les
associations, agrégations, généralisation, interfaces, etc ;\\
\textbullet Diagramme d'objets : il sert à représenter les instances de classes (objets) utilisées dans le système ;\\
\textbullet Diagramme de composants : il permet de montrer les composants du système d'un point de vue physique ; \\
\textbullet Diagramme de déploiement : il sert à représenter les éléments matériels et la manière dont les composants du système sont répartis sur ces éléments matériels et interagissent entre eux ;\\
\textbullet Diagramme de paquetages : il sert à représenter les dépendances entre paquetages, c’est-à-dire les dépendances entre ensembles de définitions ;\\
\textbullet Diagramme de structure composite : il montre l’organisation interne d’un élément statique complexe ;\\
\textbullet Diagramme de profils : il permet de spécialiser, de personnaliser pour un domaine particulier un méta-modèle de référence d'UML ;\\
\textgreater Diagrammes comportementaux :
\textbullet Diagramme des cas d'utilisation : il représente la structure des grandes fonctionnalités nécessaires aux utilisateurs du système ;\\ 
\textbullet Diagramme d'états-transitions : il représente la façon dont évoluent les objets appartenant à une même classe ; \\
\textbullet Diagramme d'activités : le diagramme d'activités (cf. section 6) n'est autre que la transcription dans UML de la représentation du processus telle qu'elle a été élaborée lors du travail qui a préparé la modélisation : il montre l'enchaînement des activités qui concourent au processus ; \\
\textgreater Diagrammes d’interaction ou dynamiques :
\textbullet Diagramme de séquence : ; \\
\textbullet Diagramme de communication : 
\textbullet Diagramme global d'interaction : ; \\
\textbullet Diagramme de temps : ; \\
Ces diagrammes, d'une utilité variable selon les cas, ne sont pas nécessairement tous produits à l'occasion d'une modélisation. Les plus utiles pour la maîtrise d'ouvrage sont les diagrammes d'activités, de cas d'utilisation, de classes, d'objets, de séquence et d'états-transitions. Les diagrammes de composants, de déploiement et de communication sont surtout utiles pour la maîtrise d'œuvre à qui ils permettent de formaliser les contraintes de la réalisation et la solution technique.
Nous utiliserons au besoin certains de ces diagrammes pour illustrer les aspects de notre solution.

\clearpage 

