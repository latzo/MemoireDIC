\chapter{Analyse et Conception}
\minitoc
\clearpage
Nous nous sommes intéressés au Top 10 d'OWASP. qui nous a servi de cahiers de charge pour notre recueil de besoins. Les spécifications ont été faites sur la base de ce document. La version originale du document est disponible en annexe.\\
Le Top 10 OWASP, en recensant les dix risques de sécurité les plus critiques des applications Web, les explique et donne pour chacun d'eux, un ensemble de directives de codage à mettre en œuvre pour se protéger de ces risques de sécurité. Cependant, le document original n'est pas assez parlant pour bon nombre de personnes et le fait qu'il soit rédigé en anglais est un obstacle pour d'autres. Nous avons tenté d'expliquer chaque point du Top 10 afin de les faire connaître aux développeurs. Cela a fait l'objet d'un document distribué aux développeurs et affiché dans les locaux de HubSo. De même, nous avons, pour chaque risque, recensé les pratiques à mettre en œuvre pour s'en prémunir. Celles-ci feront l'objet des spécifications et de l'analyse.

\section{Spécifications}

\subsection{Spécifications fonctionnelles}
Les spécifications fonctionnelles décrivent les processus métier dans lesquels notre système devra intervenir, les tâches prises en charge par le système. Dans notre cas, il s'agira des pratiques préconisées par le Top 10 pour chacun des risques de sécurité. 

\subsubsection{Les Acteurs}
Un acteur représente un rôle joué par une entité externe (utilisateur humain, dispositif matériel ou autre système) qui interagit directement avec le système étudié.\\
Notre système est principalement en interaction avec les autres applications utilisant les différentes fonctionnalités mises à disposition par celui-ci. 

\subsubsection{Les fonctionnalités générales}
Notre système met à la disposition des applications qui l'utilisent un ensemble de fonctionnalités leur permettant de gérer les différentes risques de sécurité énoncées par le Top 10. Il s'agit d'une bibliothèque de sécurité. Ces fonctionnalités, sont dans un premier regroupées en modules avec chaque module correspondant à la gestion d'un risque.\\

\textbf{\RIGHTarrow Gestion des injections}\\
Ce module regroupe les fonctions de sécurité permettant de se protéger des injections. Pour prévenir les injections, les fonctions de sécurités suivantes doivent être mises à disposition par le système : 
\begin{itemize}
	\itemcheck la validation des entrées ; 
	\itemcheck la récupération de données de manière sécurisée ; 
	\itemcheck l’encodage des données ; 
	\itemcheck le paramétrage de requêtes vers les systèmes de gestion de base de données ; 
	\itemcheck le cryptage de données avec un algorithme fort.\\
\end{itemize}

\textbf{\RIGHTarrow Gestion des violations de Gestion d’Authentification}\\
Ce module regroupe les fonctions de sécurité permettant de se protéger des violations de gestion d'authentification. Pour prévenir la violation de gestion d'authentification, les fonctions de sécurités suivantes doivent être disponibles dans notre système :
 \begin{itemize}
 	\itemcheck l'authentification ;
 	\itemcheck la déconnexion ;
 	\itemcheck la vérification de la force d'un mot de passe ; 
 	\itemcheck la génération aléatoire de données : il s'agit de générer aléatoirement avec un algorithme non prévisible des données aléatoires numériques et alphanumériques ; 
 	\itemcheck la journalisation des événements de connexion (logging) ; 
 	\itemcheck le cryptage de données avec un algorithme fort.\\
 \end{itemize}

\textbf{\RIGHTarrow Gestion des expositions de données sensibles}\\
Ce module regroupe les fonctions de sécurité permettant d'éviter l'exposition de données sensibles. Pour ce faire, les fonctions de sécurités suivantes doivent être disponibles dans notre système :
\begin{itemize}
	\itemcheck le hachage fort de mots de passe avec un sel\footnote{donnée ajoutée au mot de passe avant hachage} ;
	\itemcheck l'ajout d'entêtes HTTP (Entête Cache essentiellement) ;
	\itemcheck la vérification de l'utilisation d'un canal sécurisé HTTPS ;
	\itemcheck l'authentification des requêtes HTTP en Digest (Voir annexe) ;
	\itemcheck le cryptage de données avec un algorithme fort.\\
\end{itemize}

\textbf{\RIGHTarrow Gestion des attaques sur les entités XML externes}\\
Ce module regroupe les fonctions de sécurité permettant de se protéger des attaques sur les entités XML externes. Il contient les fonctions de sécurités suivantes :
\begin{itemize}
	\itemcheck l'encodage de données en HTML ;
	\itemcheck la validation des entrées.\\
\end{itemize}

\textbf{\RIGHTarrow Gestion des violations de contrôle d’accès}\\
Ce module regroupe les fonctions de sécurité permettant de se protéger des violations de contrôle d'accès. Pour prévenir les violations de contrôle d'accès, les fonctions de sécurités suivantes doivent être mises à disposition par le système :
\begin{itemize}
	\itemcheck la vérification des autorisations sur les ressources (autorisations sur les fichiers, urls, fonctions) ; 
	\itemcheck la vérification des rôles des utilisateurs ; 
	\itemcheck l'authentification ; 
	\itemcheck la déconnexion ; 
	\itemcheck la journalisation des accès aux ressources et évènements de connexion.\\
\end{itemize}

\textbf{\RIGHTarrow Gestion des mauvaises configurations de sécurité}\\
Ce module regroupe les fonctions de sécurité permettant d'éviter les problèmes de sécurité découlant des mauvaises configurations de sécurité. On a principalement :
\begin{itemize}
	\itemcheck le paramétrage des entêtes HTTP : il s'agit de toutes les entêtes HTTP relatives à la sécurité (Entêtes HSTS, X-Frame-Options, X-XSS-Protection, X-Content-Type-Options, CSP, X-Permitted-Cross-Domain-Policies entre autres). \\
\end{itemize}

\textbf{\RIGHTarrow Gestion des cross-site scripting (XSS)}\\
Ce module regroupe un ensemble de fonctions de sécurité permettant de se protéger des attaques XSS. Il s'agit de :
\begin{itemize}
	\itemcheck la validation des entrées ;
	\itemcheck la sanitization des entrées ;
	\itemcheck l'encodage des sorties selon le contexte de sortie (HTML, CSS, JavaScript)\footnote{Toute donnée devant être ajoutée au code source};
	\itemcheck l'ajout d'entêtes HTTP (CSP principalement).\\
\end{itemize}

\textbf{\RIGHTarrow Gestion des désérialisations non sécurisée}\\
Ce module regroupe les fonctions de sécurité permettant d'éviter les problèmes de sécurité découlant des désérialisations non sécurisées. On a principalement :
\begin{itemize}
	\itemcheck la signature d'un message ;
	\itemcheck la vérification de signatures ; 
	\itemcheck la validaation de données.\\
\end{itemize}

\textbf{\RIGHTarrow Gestion de l'utilisation de composants vulnérables}\\
Ce module regroupe les fonctions de sécurité relatives à l'utilisation de composants vulnérables. Il regroupe les fonctionnalités suivantes :
\begin{itemize}
	\itemcheck la détection de composants vulnérables ;
	\itemcheck la notification de nouvelle version d'un composant.\\
\end{itemize}

\textbf{\RIGHTarrow Gestion de la journalisation et de la surveillance insuffisante}\\
Ce module regroupe les fonctions de sécurité relatives à la journalisation et à la surveillance. Il s'agit de journaliser afin d'avoir une trace de ce qui se passe dans le système mais aussi surveiller afin d'être réactif par rapport aux évènements du système. Les fonctionnalités sont les suivantes :
\begin{itemize}
	\itemcheck la journalisation des évènements.
\end{itemize}

\subsection{Spécifications non fonctionnelles}
Les besoins non fonctionnelles ou exigences techniques portent sur les différents points suivants :
\begin{itemize}
	\itemcheck «pinner» les certificats ; 
	\itemcheck utiliser de protocoles sécurisés ;
	\itemcheck ne jamais déployer avec les credentials par défaut ;
	\itemcheck implémenter des mécanismes de vérifications de mots de passe par rapport à une politique de mots de passe prédéfinie ;
	\itemcheck supprimer ou ne pas installer des composants inutilisés.
	\itemcheck mettre en place une architecture d'application segmentée offrant une séparation efficace et sécurisée entre les composants ;
	\itemcheck envoi de directives de sécurité aux clients, par exemple des entêtes HTTP de sécurité ;
	\itemcheck toujours vérifier l'efficacité des configurations et des paramètres dans tous les environnements ;
	\itemcheck portabilité de la bibliothèque ;
	\itemcheck compatibilité de la bibliothèque avec les applications existantes ;
	\itemcheck formation sur l'utilisation ;
	\itemcheck efficacité : efficacité en temps, efficacité en ressources ;
	\itemcheck facilité d'intégration ;
	\itemcheck paramétrabilité de la bibliothèque par des fichiers texte de configuration.
\end{itemize}

\section{Analyse}
Après avoir défini les acteurs et énuméré les fonctionnalités générales du système, nous passons à la phase d'analyse. Nous utiliserons les diagrammes de cas d'utilisation pour mieux représenter ce qui est attendu du système. Pour certains cas d'utilisation, une description textuelle sera faite. Nous utiliserons aussi des diagrammes d'activité pour illustrer le séquencement des actions par rapport à certains cas d'utilisation.\\
Pour des besoins de concision, chaque point du Top 10 sera considéré comme un package et comportera les fonctions définies plus tôt. Ces packages sont les suivants :
\begin{itemize}
	\itemtirait package gestion des injections ;
	\itemtirait package gestion des violations de Gestion d’Authentification ;
	\itemtirait package gestion des expositions de données sensibles ;
	\itemtirait package gestion des attaques sur les entités XML externes ;
	\itemtirait package gestion des violations de contrôle d’accès ;
	\itemtirait package gestion des cross-site scripting (XSS) ;
	\itemtirait package gestion de l'utilisation de composants vulnérables ;
	\itemtirait package gestion de la journalisation et de la surveillance insuffisante.
\end{itemize}
La figure \ref{fig:8.1} ci-dessous représente ces différents packages.
\begin{figure}[h!]
	\centering
	\begin{minipage}{18cm}
		\centering
		{\includegraphics[height=0.27\textheight]{fig/Package-Diagram.png}}
	\end{minipage}
	\caption{Diagramme de packages du système}
	\label{fig:8.1}
\end{figure}

\subsection{Package Gestion des injections}

\subsubsection{Diagramme de cas d'utilisation}
Le diagramme de cas d'utilisation du package "Gestion des injections" est représenté ci-dessous. Il comprend les cas d'utilisation permettant à une application donnée de mitiger les risques d'injection.\\ 
\begin{figure}[H]
	\centering
	\begin{minipage}{12cm}
		\centering
		{\includegraphics[height=0.30\textheight]{fig/Injection-use-case-diagram.png}}
	\end{minipage}
	\caption{Diagramme de cas d'utilisation du package "Gestion des injections"}
	\label{fig:7.2}
\end{figure}

\subsection{Package Gestion des violations de gestion d'authentification}

\subsubsection{Diagramme de cas d'utilisation}
Le diagramme de cas d'utilisation du package "Gestion des violations de gestion d'authentification" est représenté ci-dessous. Il comprend les cas d'utilisation permettant à une application donnée de mitiger les risques de violations de gestion d'authentification.\\ 
\begin{figure}[H]
	\centering
	\begin{minipage}{12cm}
		\centering
		{\includegraphics[height=0.30\textheight]{fig/Violation-gestion-authentification-use-case-diagram.png}}
	\end{minipage}
	\caption{Diagramme de cas d'utilisation du package "Gestion des violations de gestion d'authentification"}
	\label{fig:7.3}
\end{figure}

\subsection{Package Gestion des expositions de données sensibles}

\subsubsection{Diagramme de cas d'utilisation}
Le diagramme de cas d'utilisation du package "Gestion des expositions de données sensibles" est representé ci-dessous. Il comprend les cas d'utilisation permettant à une application donnée de mitiger les risques d'exposition de données sensibles.\\ 
\begin{figure}[H]
	\centering
	\begin{minipage}{12cm}
		\centering
		{\includegraphics[height=0.30\textheight]{fig/Gestion-exposition-donnees-use-case-diagram.png}}
	\end{minipage}
	\caption{Diagramme de cas d'utilisation du package "Gestion des expositions de données sensibles"}
	\label{fig:7.4}
\end{figure}

\subsection{Package Gestion des attaques sur les entités XML externes}
\subsubsection{Diagramme de cas d'utilisation}
Le diagramme de cas d'utilisation du package "Gestion des attaques sur les entités XML externes" est representé ci-dessous. Il comprend les cas d'utilisation permettant à une application donnée de mitiger les risques découlant des attaques sur les entités XML externes.\\ 
\begin{figure}[H]
	\centering
	\begin{minipage}{12cm}
		\centering
		{\includegraphics[height=0.30\textheight]{fig/XXE-use-case-diagram.png}}
	\end{minipage}
	\caption{Diagramme de cas d'utilisation du package "Gestion des attaques sur les entités XML externes"}
	\label{fig:7.5}
\end{figure}

\subsection{Package Gestion des violations de contrôle d'accès}
\subsubsection{Diagramme de cas d'utilisation}
Le diagramme de cas d'utilisation du package "Gestion des violations de contrôle d'accès" est représenté ci-dessous. Il comprend les cas d'utilisation permettant à une application donnée de mitiger les risques découlant des violations de contrôle d'accès.\\ 
\begin{figure}[H]
	\centering
	\begin{minipage}{12cm}
		\centering
		{\includegraphics[height=0.30\textheight]{fig/Violation-controle-acces-use-case-diagram.png}}
	\end{minipage}
	\caption{Diagramme de cas d'utilisation du package "Gestion des violations de contrôle d'accès"}
	\label{fig:7.6}
\end{figure}

\subsection{Package Gestion des mauvaises configurations de sécurité}
\subsubsection{Diagramme de cas d'utilisation}
Le diagramme de cas d'utilisation du package "Gestion des mauvaises configurations de sécurité" est représenté ci-dessous. Il comprend les cas d'utilisation permettant à une application donnée de mitiger les risques de mauvaises configurations de sécurité.\\ 
\begin{figure}[H]
	\centering
	\begin{minipage}{12cm}
		\centering
		{\includegraphics[height=0.30\textheight]{fig/Security_misconfiguration-use-case-diagram.png}}
	\end{minipage}
	\caption{Diagramme de cas d'utilisation du package "Gestion des mauvaises configurations de sécurité"}
	\label{fig:7.7}
\end{figure}

\subsection{Package Gestion des XSS }
\subsubsection{Diagramme de cas d'utilisation}
Le diagramme de cas d'utilisation du package "Gestion des XSS" est représenté ci-dessous. Il comprend les cas d'utilisation permettant à une application donnée de mitiger les attaques XSS.\\ 
\begin{figure}[H]
	\centering
	\begin{minipage}{12cm}
		\centering
		{\includegraphics[height=0.30\textheight]{fig/XSS-use-case-diagram.png}}
	\end{minipage}
	\caption{Diagramme de cas d'utilisation du package "Gestion des XSS"}
	\label{fig:7.8}
\end{figure}

\subsection{Package Gestion des désérialisations non sécurisées}
\subsubsection{Diagramme de cas d'utilisation}
Le diagramme de cas d'utilisation du package "Gestion des désérialisations non sécurisées" est représenté ci-dessous. Il comprend les cas d'utilisation permettant à une application donnée de mitiger les risques de désérialisations non sécurisées.\\ 
\begin{figure}[H]
	\centering
	\begin{minipage}{12cm}
		\centering
		{\includegraphics[height=0.25\textheight]{fig/Insecure_deserialization-use-case-diagram.png}}
	\end{minipage}
	\caption{Diagramme de cas d'utilisation du package "Gestion des désérialisations non sécurisées"}
	\label{fig:7.9}
\end{figure}

\subsection{Package Gestion des utilisations de composants vulnérables}
\subsubsection{Diagramme de cas d'utilisation}
Le diagramme de cas d'utilisation du package "Gestion des utilisations de composants vulnérables" est représenté ci-dessous. Il comprend les cas d'utilisation permettant à une application donnée de mitiger les risques découlant de l'utilisation de composants vulnérables.\\ 
\begin{figure}[H]
	\centering
	\begin{minipage}{12cm}
		\centering
		{\includegraphics[height=0.30\textheight]{fig/Utilisation-composants-vulnerables-use-case-diagram.png}}
	\end{minipage}
	\caption{Diagramme de cas d'utilisation du package "Gestion des utilisations de composants vulnérables"}
	\label{fig:7.10}
\end{figure}

\subsection{Package Gestion de la journalisation et de la surveillance insuffisantes}
\subsubsection{Diagramme de cas d'utilisation}
Le diagramme de cas d'utilisation du package "Gestion de la journalisation et de la surveillance insuffisantes" est représenté ci-dessous. Il comprend les cas d'utilisation permettant à une application donnée de mitiger les risques de journalisation et de surveillance insuffisantes.\\ 
\begin{figure}[H]
	\centering
	\begin{minipage}{12cm}
		\centering
		{\includegraphics[height=0.30\textheight]{fig/Insufficient-logging-monitoring-use-case-diagram.png}}
	\end{minipage}
	\caption{Diagramme de cas d'utilisation du package "Gestion de la journalisation et de la surveillance insuffisantes"}
	\label{fig:7.11}
\end{figure}

\subsection{Système global}
Nous avons organisé les cas d'utilisation en packages, chaque package correspondant à la gestion d'un risque du Top 10, et cela pour des questions de lisibilité mais aussi pour des questions d'identification des fonctions de sécurité à mettre en place en adéquation avec le risque en question. Cependant, le système que nous devons mettre en place tourne autour de ces fonctions de sécurité. Il s'agit de mettre à la disposition des développeurs ces fonctions de sécurités. \\
En effet, certaines fonctions de sécurité recommandées pour un risque, peuvent aussi l'être pour d'autres.Par exemple, si l'on essaie de mitiger le risque de XSS, la meilleure façon de le faire est de mettre en place des fonctions permettant la validation des entrées ainsi que l'encodage des sorties que les développeurs peuvent facilement utiliser. Mais ces mêmes fonctions peuvent être utilisées pour se protéger de beaucoup  d’autres attaques. \\
Ainsi, nous nous concentrons maintenant sur ces fonctions de sécurité. Nous présenterons à la volée toutes les fonctionnalités attendues par les applications pour se protéger au moins des dix risques de sécurité présentes dans le Top 10.\\
Ci-dessous, nous avons le diagramme de cas d'utilisation global du système :
\begin{figure}[H]
	\centering
	\includegraphics[width=1.0\textwidth,height=0.95\textheight]{fig/Global-use-case-diagram.png}
	\caption{Diagramme de cas d'utilisation global du système}
	\label{fig:7.12}
\end{figure}
Toutefois, cette représentation n'est pas non plus la meilleure car ne favorisant pas une bonne lisibilité. Ainsi, nous avons décidé pour des raisons de lisibilité de séparer ces fonctions en modules, chaque module comprenant les fonctions de sécurité de même nature.\\
Ainsi, nous avons les modules suivants :
\begin{itemize}
	\itemcheck Module "Utilisateurs" comprenant les fonctions relatives aux utilisateurs ;
	\itemcheck Module "Cryptographie" comprenant les fonctions se rapportant à la cryptographie ; 
	\itemcheck Module "Encodage" comprenant les fonctions d'encodage ; 
	\itemcheck Module "Validation" regroupant les fonctions relatives à la validation ; 
	\itemcheck Module "HTTP" regroupant les fonctions relatives aux paramètres HTTP ;
	\itemcheck Module "Interpréteurs" regroupant les fonctions relatives aux interpréteurs ;
	\itemcheck Module "Logging" regroupant les fonctions relatives à la journalisation ; 
	\itemcheck Module "Gestion des composants" regroupant les fonctions relatives aux composants utilisés dans l'application.
\end{itemize}
Chaque module est un sous-sytème et sera représenté par un package. Cette séparation nous donne le diagramme de packages suivant :
\begin{figure}[H]
	\centering
	\includegraphics[height=0.5\textheight]{fig/S-Package-diagram.png}
	\caption{Diagramme de packages du système (réorganisé)"}
\end{figure}


\subsection{Sous-système "Utilisateurs"}

\subsubsection{Diagramme de cas d'utilisation}
\begin{figure}[H]
	\centering
	\begin{minipage}{12cm}
		\centering
		{\includegraphics[height=0.35\textheight, width=1\textwidth]{fig/Utilisateurs-use-case-diagram.png}}
	\end{minipage}
	\caption{Diagramme de cas d'utilisation du sous-système "Utilisateurs"}
	\label{fig:7.13}
\end{figure}

\subsubsection{Cas d'utilisation "Authentification utilisateur"}
\textbf{\RIGHTarrow Description textuelle}\\
\underline{\underline{Sommaire d’identification}} \\
\textbf{Titre} : Authentification utilisateur\\
\textbf{Résumé} : Ce cas d’utilisation permet à une application d'authentifier un utilisateur.\\
\textbf{Acteur} : Application\\	
\textbf{Responsable} : Papa Latyr Mbodj\\
\underline{\underline{Description des scénarios}}\\
\textbf{Précondition(s)}
\begin{itemize}
	\item Aucune
\end{itemize}
\textbf{Scénario nominal}
\begin{enumerate}
	\item L'application donne une requête HTTP POST contenant nom d'utilisateur et mot de passe.
	\item Le système récupère l'utilisateur avec le nom d'utilisateur donné.
	\item Le système enregistre l'adresse IP de connexion.
	\item Le système s'assure que la requête est de type POST et que HTTPS est utilisé.
	\item Le système s'assure que l'utilisateur n'est pas expiré.
	\item Le système s'assure que l'utilisateur est activé.
	\item Le système s'assure que l'utilisateur n'est pas bloqué.
	%\item Le système s'assure que le délai d'inactivité de l'utilisateur n'est pas atteint.
	%\item Le système s'assure que la session de l'utilisateur n'est pas expirée.
	\item Le système déconnecte l'utilisateur.
	\item Le système hashe le mot de passe donné et le compare avec celui de l'utilisateur récupéré précédemment.
	\item Le système crée une nouvelle session pour l'utilisateur et l'enregistre.
	\item Le système enregistre la date et l'heure de connexion 
	\item Le système enregistre l'adresse IP de l'hôte ayant envoyé la requête.
	Le système met à jour l'état de connexion de l'utilisateur
	\item Le système journalise la connexion de l'utilisateur.
	\item Le système retourne l'utilisateur modifié.
\end{enumerate}
%\textbf{Enchainements alternatif(s)}\\
\textbf{Enchainements d’erreur}\\
\textit{E1 : Utilisateur avec nom d'utilisateur donné inexistant}\\
L’enchaînement E1 démarre au point 2 du scénario nominal.
\begin{itemize}
	\item[3.] Le système journalise l'échec de la connexion avec le message "Login ou mot de passe incorrect".
	\item[4.] Le système produit une exception avec le même message.
	\item[5.] Le système refuse la connexion; le cas d'utilisation se termine en échec.
\end{itemize}
\textit{E2 : Requête pas de type POST ou HTTPS non utilisé}\\
L’enchaînement E2 démarre au point 4 du scénario nominal.
\begin{itemize}
	\item[5.] Le système met à jour la date et l'heure de la dernière connexion échouée de l'utilisateur.
	\item[6.] Le système incrémente le nombre de tentatives de connexion échouées pour cet utilisateur.
	\begin{itemize}
		\item[6.a.] Nombre maximal de tentatives de connexions échouées atteint ou dépassé : 
		\begin{itemize}
			\item[6.a.1.] Le système bloque l'utilisateur.
			\item[6.a.2.] Le système journalise le blocage de l'utilisateur avec le message "Utilisateur bloqué" avec le nom de l'utilisateur.
			\item[6.a.3.] Le système produit une exception avec le même message.
		\end{itemize}	
		\item[6.b.] Nombre maximal de tentatives de connexions échouées non atteint :
		\begin{itemize}
			\item[6.b.1] Le système journalise l'échec de la connexion avec le message "Tentative de connexion avec une requête non sécurisée".
			\item[6.b.2] Le système produit une exception avec le même message.
		\end{itemize}	
	\end{itemize}
	\item[7.] Le système refuse la connexion; le cas d'utilisation se termine en échec.
\end{itemize}
\textit{E3 : Utilisateur expiré}\\
L’enchaînement E3 démarre au point 5 du scénario nominal.
\begin{itemize}
	\item[6.] Le système journalise l'échec de la connexion avec le message "Utilisateur expiré" avec le nom de l'utilisateur.
	\item[7.] Le système produit une exception avec le même message.
	\item[8.] Le système refuse la connexion; le cas d'utilisation se termine en échec.
\end{itemize}
\textit{E4 : Utilisateur non activé}\\
L’enchaînement E4 démarre au point 6 du scénario nominal.
\begin{itemize}
	\item[7.] Le système journalise l'échec de la connexion avec le message "Utilisateur inactif" avec le nom de l'utilisateur.
	\item[8.] Le système produit une exception avec le même message.
	\item[9.] Le système refuse la connexion; le cas d'utilisation se termine en échec.
\end{itemize}
\textit{E5 : Utilisateur bloqué}\\
L’enchaînement E5 démarre au point 7 du scénario nominal.
\begin{itemize}
	\item[8.] Le système met à jour la date et l'heure de la dernière connexion échouée de l'utilisateur.
	\item[9.] Le système incrémente le nombre de tentatives de connexion échouées pour cet utilisateur.
	\begin{itemize}
		\item[10.a.] Nombre maximal de tentatives de connexions échouées atteint ou dépassé : 
		\begin{itemize}
			\item[10.a.1.] Le système bloque l'utilisateur.
			\item[10.a.2.] Le système journalise le blocage de l'utilisateur avec le message "Utilisateur bloqué" avec le nom de l'utilisateur.
			\item[10.a.3.] Le système produit une exception avec le même message.
			\item[10.a.4.] Le système refuse la connexion; le cas d'utilisation se termine en échec.
		\end{itemize}
	\end{itemize}
\end{itemize}
%\textit{E6 : Délai d'inactivité de l'utilisateur atteint}\\
%L’enchaînement E6 démarre au point 8 du scénario nominal.
%\begin{itemize}
%	\item[9.] Le système met à jour la date et l'heure de dernière connexion échouée de l'utilisateur.
%	\item[10.] Le système déconnecte l'utilisateur.
%\item[11.] Le système journalise l'échec de la connexion avec le message "Délai d'inactivité atteint" avec le nom de l'utilisateur.
%	\item[12.] Le système produit une exception avec le même message.
%	\item[13.] Le système refuse la connexion; le cas d'utilisation se termine en échec.
%\end{itemize}
%\textit{E7 : Session utilisateur expirée}\\
%L’enchaînement E7 démarre au point 9 du scénario nominal.
%\begin{itemize}
%	\item[8.] Le système met à jour la date et l'heure de dernière connexion échouée de l'utilisateur.
%	\item[9.] Le système déconnecte l'utilisateur.
%	\item[10.] Le système journalise l'échec de la connexion avec le message "Session expirée" avec le nom de l'utilisateur.
%	\item[11.] Le système produit une exception avec le même message.
%	\item[12.] Le système refuse la connexion; le cas d'utilisation se termine en échec.
%\end{itemize}
\textit{E8 : Mot de passe erroné}\\
L’enchaînement E8 démarre au point 9 du scénario nominal.
\begin{itemize}
	\item[12.] Le système met à jour la date et l'heure de dernière connexion échouée de l'utilisateur.
	\item[13.] Le système incrémente le nombre de tentatives de connexion échouées pour cet utilisateur.
	\begin{itemize}
			\item[13.a.] Nombre maximal de tentatives de connexions échouées atteint :
			\begin{itemize}
				\item[13.a.1.] Le système bloque l'utilisateur.
				\item[13.a.2.] Le système journalise le blocage de l'utilisateur avec le message "Utilisateur bloqué" avec le nom de l'utilisateur.
				\item[13.a.3.] Le système produit une exception avec le même message.
				\item[13.a.4.] Le système refuse la connexion; le cas d'utilisation se termine en échec.
			\end{itemize}
			\item[13.b.] Nombre maximal de tentatives de connexions échouées non atteint :
			\begin{itemize}
				\item[13.b.1] Le système journalise l'échec de la connexion avec le message "Login ou mot de passe incorrect".
				\item[13.b.2] Le système produit une exception avec le même message. en échec.
			\end{itemize}
	\end{itemize}
	\item[14.] Le système refuse la connexion; le cas d'utilisation se termine en échec.
\end{itemize}
\textbf{Post-condition(s)}
\begin{itemize}
	\item L’utilisateur en question est authentifié sur l'application.
\end{itemize}
\underline{\underline{Exigences non fonctionnelles}}
\begin{itemize}
	\item L’application doit être hébergée sur un serveur protégé par un firewall anti DDoS.\\
\end{itemize}

\textbf{\RIGHTarrow Diagramme d'activités}\\
\begin{figure}[H]
	\centering
	\includegraphics[width=1.0\textwidth,height=0.95\textheight]{fig/Authentification-utilisateur-activity-diagram.png}
	\caption{Diagramme d'activités du cas "Authentification utilisateur"}
\end{figure}

\subsubsection{Cas d'utilisation "Déconnexion utilisateur"}
\textbf{\RIGHTarrow Description textuelle}\\
\underline{\underline{Sommaire d’identification}} \\
\textbf{Titre} : Déconnexion utilisateur\\
\textbf{Résumé} : Ce cas d’utilisation permet à une application de déconnecter un utilisateur.\\
\textbf{Acteur} : Application\\	
\textbf{Responsable} : Papa Latyr Mbodj\\
\underline{\underline{Description des scénarios}}\\
\textbf{Précondition(s)}
\begin{itemize}
	\item Aucune
\end{itemize}
\textbf{Scénario nominal}
\begin{enumerate}
	\item Le système supprime le cookie "Se rappeler de moi".
	\item Le système invalide la session contenue dans la requête courante.
	\item Le système supprime le cookie contenant l'identificateur de session.
	\item Le système met à jour l'état de connexion de l'utilisateur.
	\item Le système journalise la déconnexion de l'utilisateur.
\end{enumerate}
%\textbf{Enchainements alternatif(s)}\\
\textbf{Post-condition(s)}
\begin{itemize}
	\item L’utilisateur en question est déconnecté de l'application.
\end{itemize}

\subsubsection{Cas d'utilisation "Vérification force mot de passe"}
\textbf{\RIGHTarrow Description textuelle}\\
\underline{\underline{Sommaire d’identification}} \\
\textbf{Titre} : Vérification force mot de passe\\
\textbf{Résumé} : Ce cas d’utilisation permet à une application de vérifier qu'un nouveau mot de passe est fort.\\
\textbf{Acteur} : Application\\	
\textbf{Responsable} : Papa Latyr Mbodj\\
\underline{\underline{Description des scénarios}}\\
\textbf{Précondition(s)}
\begin{itemize}
	\item Aucune
\end{itemize}
\textbf{Scénario nominal}
\begin{enumerate}
	\item L'application donne le login, l'ancien mot de passe et le nouveau mot de passe.
	\item Le système s'assure que le nouveau mot de passe n'est pas nul.
	\item Le système s'assure que le nouveau mot de passe ne correspond pas au login.
	\item Le système s'assure que le nouveau mot de passe ne contient pas une partie de l'ancien mot de passe (3 caractères).
	\item Le système compte le nombre de lettres minuscules contenues dans le nouveau mot de passe.
	\item Le système compte le nombre de lettres majuscules contenues dans le nouveau mot de passe.
	\item Le système compte le nombre de chiffres contenus dans le nouveau mot de passe.
	\item Le système compte le nombre de caractères spéciaux contenus dans le nouveau mot de passe.
	\item Le système évalue la force du mot de passe.
	\item Le mot de passe est accepté.
\end{enumerate}
%\textbf{Enchainements alternatif(s)}\\
\textbf{Enchainements d’erreur}\\
\textit{E1 : Nouveau mot de passe nul}\\
L’enchaînement E1 démarre au point 2 du scénario nominal.
\begin{itemize}
	\item[3.] Le système journalise l'invalidité du mot de passe avec le message "Le nouveau mot de passe ne peut pas être nul".
	\item[4.] Le système produit une exception avec le même message.
	\item[5.] Le système refuse le mot de passe; le cas d'utilisation se termine en échec.
\end{itemize}
\textit{E2 : Nouveau mot de passe correspondant au login}\\
L’enchaînement E2 démarre au point 3 du scénario nominal.
\begin{itemize}
	\item[4.] Le système journalise l'invalidité du mot de passe avec le message "Le nouveau mot de passe ne peut pas correspondre au login".
	\item[5.] Le système produit une exception avec le même message.
	\item[6.] Le système refuse le mot de passe; le cas d'utilisation se termine en échec.
\end{itemize}
\textit{E3 : Nouveau mot de passe contenant une partie de l'ancien mot de passe}\\
L’enchaînement E3 démarre au point 4 du scénario nominal.
\begin{itemize}
	\item[5.] Le système journalise l'invalidité du mot de passe avec le message "Le nouveau mot de passe ne peut pas contenir une partie de l'ancien mot de passe".
	\item[6.] Le système produit une exception avec le même message.
	\item[7.] Le système refuse le mot de passe; le cas d'utilisation se termine en échec.
\end{itemize}
\textit{E4 : Mot de passe faible}\\
L’enchaînement E4 démarre au point 9 du scénario nominal.
\begin{itemize}
	\item[10.] Le système journalise l'invalidité du mot de passe avec le message "Le nouveau mot de passe n'est pas long ou non complexe".
	\item[11.] Le système produit une exception avec le même message.
	\item[12.] Le système refuse le mot de passe; le cas d'utilisation se termine en échec.
\end{itemize}
\textbf{Post-condition(s)}
\begin{itemize}
	\item Le nouveau mot de passe est accepté.
\end{itemize}
\underline{\underline{Exigences non fonctionnelles}}
\begin{itemize}
	\item L’application doit définir une bonne politique de mots de passe.\\
\end{itemize}
\textbf{\RIGHTarrow Diagramme d'activités}\\
La force du mot de passe est calculé en considérant que la taille minimale d'un mot de passe est de 4 caractères. Et pour que ce mot de passe soit fort, il faudrait qu'il contienne 1 lettre minuscule, 1 lettre majuscule, 1 chiffre et 1 caractère spécial. Sa force est de (1+1+1+1) * 4 (nombre de caractères du mot de passe) ce qui donne 16. La force minimale est ainsi de 16.
\begin{figure}[H]
	\centering
	\includegraphics[width=1.0\textwidth,height=0.95\textheight]{fig/Verification-force-mot-de-passe-activity-diagram.png}
	\caption{Diagramme d'activités du cas "Authentification utilisateur"}
\end{figure}

\subsubsection{États d'un utilisateur}
\begin{figure}[H]
	\centering
	\begin{minipage}{12cm}
		\centering
		{\includegraphics[height=0.35\textheight, width=1\textwidth]{fig/Users-statemachine-diiagram.png}}
	\end{minipage}
	\caption{Diagramme d'états-transition d'un utilisateur}
	\label{fig:7.13}
\end{figure}
Lorsqu'un utilisateur est nouvellement créé, il se doit d'activer son compte. S'il ne le fait pas au bout d'un certain délai, l'utilisateur nouvellement créé \textit{expire} et il n'a plus la possibilité de se connecter à l'application. Après activation, l'utilisateur est \textit{activé} peut se connecter. Après une connexion réussie, l'utilisateur est \textit{connecté}. Lorsque connecté, une déconnexion, un délai de session écoulé ou une inactivité au bout d'un certain délai, cet utilisateur est \textit{déconnecté} et seule une nouvelle authentification réussie lui permet d'être \textit{connecté} à nouveau. Lors de l'authentification, après un certain nombre de tentatives de connexion échouées préalablement fixé, l'utilisateur est \textit{bloqué} et seul le déblocage de l'utilisateur par un administrateur peut le remettre à un état \textit{débloqué}

\subsection{Sous-système "Cryptographie"}
\subsubsection{Diagramme de cas d'utilisation}
\begin{figure}[H]
	\centering
	\begin{minipage}{12cm}
		\centering
		{\includegraphics[height=0.35\textheight, width=1\textwidth]{fig/Cryptographie-use-case-diagram.png}}
	\end{minipage}
	\caption{Diagramme de cas d'utilisation du sous-système "Cryptographie"}
	\label{fig:7.14}
\end{figure}
%\subsubsection{Cas d'utilisation "Crypter sans clé"}
*%\textbf{\RIGHTarrow Description textuelle}\\
%\textbf{\RIGHTarrow Diagramme d'activités}\\
%\subsubsection{Cas d'utilisation "Hasher sans sel"}
%\textbf{\RIGHTarrow Description textuelle}\\
%\textbf{\RIGHTarrow Diagramme d'activités}\\

\subsection{Sous-système "Encodage"}
\subsubsection{Diagramme de cas d'utilisation}
\begin{figure}[H]
	\centering
	\begin{minipage}{12cm}
		\centering
		{\includegraphics[height=0.35\textheight, width=1\textwidth]{fig/Encodage-use-case-diagram.png}}
	\end{minipage}
	\caption{Diagramme de cas d'utilisation du sous-système "Encodage"}
	\label{fig:7.15}
\end{figure}

\subsection{Sous-système "Validation"}
\subsubsection{Diagramme de cas d'utilisation}
\begin{figure}[H]
	\centering
	\begin{minipage}{12cm}
		\centering
		{\includegraphics[height=0.35\textheight, width=1\textwidth]{fig/Validation-use-case-diagram.png}}
	\end{minipage}
	\caption{Diagramme de cas d'utilisation du sous-système "Validation"}
	\label{fig:7.16}
\end{figure}
%TODO \subsubsection{Cas d'utilisation "Valider donnée"}
%TODO \textbf{\RIGHTarrow Description textuelle}\\
%TODO \textbf{\RIGHTarrow Diagramme d'activités}\\

\subsection{Sous-système "HTTP"}
\subsubsection{Diagramme de cas d'utilisation}
\begin{figure}[H]
	\centering
	\begin{minipage}{12cm}
		\centering
		{\includegraphics[height=0.35\textheight, width=1\textwidth]{fig/HTTP-use-case-diagram.png}}
	\end{minipage}
	\caption{Diagramme de cas d'utilisation du sous-système "HTTP"}
	\label{fig:7.17}
\end{figure}
%TODO \subsubsection{Cas d'utilisation "Ajouter cookie"}
%TODO \textbf{\RIGHTarrow Description textuelle}\\
%TODO \textbf{\RIGHTarrow Diagramme d'activités}\\

\subsection{Sous-système "Interpréteurs"}
\subsubsection{Diagramme de cas d'utilisation}
\begin{figure}[H]
	\centering
	\begin{minipage}{12cm}
		\centering
		{\includegraphics[height=0.3\textheight, width=1\textwidth]{fig/Interpreteurs-use-case-diagram.png}}
	\end{minipage}
	\caption{Diagramme de cas d'utilisation du sous-système "Interpreteurs"}
	\label{fig:7.18}
\end{figure}


\subsection{Sous-système "Logging"}
\subsubsection{Diagramme de cas d'utilisation}
\begin{figure}[H]
	\centering
	\begin{minipage}{12cm}
		\centering
		{\includegraphics[height=0.35\textheight, width=1\textwidth]{fig/Logging-use-case-diagram.png}}
	\end{minipage}
	\caption{Diagramme de cas d'utilisation du sous-système "Gestion de la journalisation et de la surveillance insuffisantes"}
	\label{fig:7.19}
\end{figure}

\subsection{Sous-système "Gestion des composants"}
\subsubsection{Diagramme de cas d'utilisation}
\begin{figure}[H]
	\centering
	\begin{minipage}{12cm}
		\centering
		{\includegraphics[height=0.35\textheight, width=1\textwidth]{fig/Composants-use-case-diagram.png}}
	\end{minipage}
	\caption{Diagramme de cas d'utilisation du sous-système "Gestion de la journalisation et de la surveillance insuffisantes"}
	\label{fig:7.20}
\end{figure}
%\subsubsection{Cas d'utilisation "Détecter composants vulnérables"}
%\textbf{\RIGHTarrow Description textuelle}\\
%\textbf{\RIGHTarrow Diagramme d'activités}\\
\subsection{Structure statique du système}
\subsubsection{Diagramme de classes d'analyse}
\begin{figure}[H]
	\centering
	\begin{minipage}{12cm}
		\centering
		{\includegraphics[height=0.5\textheight, width=1.2\textwidth]{fig/Analyse-Class-Diagram.png}}
	\end{minipage}
	\caption{Diagramme de classes d'analyse du système}
	\label{fig:7.20}
\end{figure}

\section{Conception}
\subsection{Synthèse de la solution}
Notre solution consiste en la mise en place d'une bibliothèque de fonctions de sécurité permettant aux applications de se protéger au moins des risques énoncés dans Top 10 OWASP 2017. Cependant, il faut le dire, ces fonctions peuvent aussi permettre de se protéger d'autres risques qui ne sont pas énoncés dans le Top 10. Les modules fonctionnels sont les suivants :
\begin{itemize}
	\itemcheck Module "Utilisateurs" ;
	\itemcheck Module "Cryptographie" ; 
	\itemcheck Module "Encodage" ; 
	\itemcheck Module "Validation" ; 
	\itemcheck Module "HTTP" ;
	\itemcheck Module "Interpréteurs" regroupant les fonctions relatives aux interpréteur
	\itemcheck Module "Logging" regroupant les fonctions relatives à la journalisation ; 
	\itemcheck Module "Gestion des composants".
\end{itemize}
Pour ce faire, nous allons utiliser une bibliothèque de fonctions de sécurité déjà existante mise à disposition par OWASP, OWASP ESAPI pour mettre en place notre propre bibliothèque de sécurité que nous appelons HubSo ESAPI.
\subsection{Utilisation de la bibliothèque OWASP ESAPI}
Comme nous l'avons dit précédemment, nous utiliserons la bibliothèque OWASP ESAPI qui contient déjà presque toutes les fonctions de sécurité énoncées dans nos différents modules. Ce choix a été fait à plusieurs égards :
\begin{itemize}
	\itemtirait L'implémentation de fonctions de sécurité prend du temps et est extrêmement sujette aux erreurs. Le projet CWE de MITRE répertorie plus de 600 types d’erreurs de sécurité que les développeurs peuvent commettre, et la plupart d’entre elles ne sont pas évidentes. Il est couramment accepté que les développeurs ne doivent pas créer leurs propres mécanismes de chiffrement, mais le même argument s'applique aux autres fonctions de sécurité.
	\itemtirait Il existe de nombreuses bibliothèques offrant diverses fonctions de sécurité: Log4j, JCE (Java Cryptographic Extension), JAAS, Acegi entre autres. Certains d'entre eux sont même très bons. Cependant, il y a plusieurs raisons faisant qu'il n'est pas idéal de les utiliser directement. La plus importante est que la majorité ces bibliothèques sont trop puissantes. La plupart des développeurs n’ont besoin que d’un ensemble très limité de fonctions de sécurité et n’ont pas besoin d’une interface complexe. En outre, beaucoup de ces bibliothèques contiennent elles-mêmes des failles de sécurité.
	\itemtirait OWASP ESAPI est avant tout un ensemble d'interfaces conçues pour faciliter la mise en place de la sécurité dans une application. Ces interfaces sont utilisables par quiconque se soucie de les implémenter par rapport à sa propre organisation. En maintenant les interfaces séparées, chacun peut produire sa propre implémentation. Cependant, OWASP ne s'est pas arrêté là : il y a déjà une implémentation de référence complète et bien testée.
	\itemtirait Même si l'on n'est pas adepte de l'opensource, toute entreprise rigoureuse devrait envisager de mettre en place une bibliothèque de sécurité pour ses développeur. Avec le projet OWASP comme modèle, cette tâche devient simple. On peut adopter uniquement les interfaces ESAPI et utiliser des parties de l’implémentation de référence.
\end{itemize}
\subsection{Conception Architecturale}
\subsubsection{Architecture Générique}
\begin{figure}[H]
	\centering
	\begin{minipage}{12cm}
		\centering
		\includegraphics[width=0.7\linewidth]{parts/part04/chapters/conception/fig/ArchitectureFonctionnelleHubSoESAPI}
	\end{minipage}
		\caption{Architecture fonctionnelle Générique de HubSo ESAPI}
		\label{fig:architecturefonctionnellehubsoesapi}
\end{figure}
\subsubsection{Architecture détaillée HubSo ESAPI croisée au Top 10 OWASP 2017}
\begin{figure}[H]
	\centering
	\begin{minipage}{12cm}
		\centering
		\includegraphics[width=0.7\linewidth,height=0.25\textheight]{parts/part04/chapters/conception/fig/EsapiXtop10}
	\end{minipage}
	\caption{Architecture détaillée HubSo ESAPI croisée au Top 10 OWASP 2017}
	\label{fig:architecturedet}
\end{figure}
\subsubsection{Architecture technique d'une application sécurisée}
\begin{figure}[H]
	\centering
	\begin{minipage}{12cm}
		\centering
		\includegraphics[width=1\linewidth,height=0.90\textheight]{parts/part04/chapters/conception/fig/ArchitectureHubsoWebApplications.png}
	\end{minipage}
	\caption{Exemple d'architecture technique sécurisée d'une application}
	\label{fig:architecturedet}
\end{figure}
\subsubsection{Architecture fonctionnelle d'une application sécurisée}
\begin{figure}[H]
	\centering
	\begin{minipage}{12cm}
		\centering
		\includegraphics[width=0.7\linewidth]{parts/part04/chapters/conception/fig/archi.jpg}
	\end{minipage}
	\caption{Architecture fonctionnelle d'une application sécurisée utilisant HubSo ESAPI}
	\label{fig:architecturefonctionnellehubsoesapi}
\end{figure}

%TODO \subsection{Diagramme de classes de conception}

%TODO \subsection{Déploiement}
%La figure suivante représente le diagramme de déploiement d'une application donnée utilisant notre %bibliothèque de sécurité.
