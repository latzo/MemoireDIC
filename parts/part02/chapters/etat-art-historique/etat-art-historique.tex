\chapter{Historique}
%%\label{chap:intro}

\section{Genèse}
Le domaine de la sécurité des informations est très ancien. Déjà, dès l’antiquité, des techniques de chiffrement de l’information étaient utilisées. Vers 1900 av JC, le scribe de Khnumhotep II retraçait la vie de son maître dans sa tombe en utilisant un certain nombre de symboles inhabituels pour masquer le sens des inscriptions avec les hiéroglyphes qu’il dessinait. Vers 500 av. J.-C., les Spartiates ont développé un dispositif appelé Scytale, qui a été utilisé pour envoyer et recevoir des messages secrets. Plus récemment dans l’histoire, l’empereur romain Jules César utilisa la technique de chiffrement qui porte son nom (Chiffrement de César) afin de crypter ses messages personnels. Ces Expériences, Bien Que Ne Couvrant Que Le Principe De La Confidentialité, Sont Les Ancêtres De La Sécurité De L’information.\\
A cette époque, assurer la sécurité de l’information se résumait en un problème de cryptage de données. On utilisait alors la cryptographie classique avec le chiffrement par substitution[1] d’abord  puis plus tard le chiffrement par transposition[2].

\section{Seconde Guerre mondiale et guerre froide : tournant de l'histoire de la sécurité informatique}
Avec le temps, les techniques permettant d’assurer la sécurité de l’information deviennent de plus en plus pointues. La seconde guerre mondiale et la guerre froide ont marqué un tournant dans l’histoire de nombreuses technologies, y compris celles qui ont façonné l’industrie de la sécurité de l’information. \\
En effet, durant cette période, de nouveaux moyens de communication apparaissent : la radio et le cinéma étaient les principaux vecteurs de l’information. Sur le champ de bataille, les différentes unités devaient coordonner leurs actions et pour ce faire des informations étaient échangées entre elles. Toutes ces communications étaient diffusées par radio et pouvaient être interceptées par l’ennemi. Il était d’une importance cruciale de rendre l’information inintelligible à l’ennemi puisqu’il s’agissait de communiquer sur les stratégies d’actions à mener. Pour protéger ces communications militaires, des systèmes très sophistiquées furent mises en place. Il s’agissait surtout de machines permettant de chiffrer l’information. Enigma était la machine de chiffrement la plus avancée de l’époque : elle sécurisait  les communications des flottes et des troupes nazis en leur permettant de chiffrer leurs messages par un chiffrement par substitution. Elle utilisait des algorithmes de chiffrement par substitution très poussés à l’époque. Elle avait la réputation d'être inviolable. Cependant, des experts en chiffrement polonais et britanniques ont réussi à trouver le moyen de casser Enigma en créant une machine connue sous le nom de « Bombe cryptographique » permettant de déchiffrer ses messages, donnant ainsi à la coalition antihitlérienne un avantage significatif  ou « l’avantage définitif » selon Churchill et Eisenhower  lors de la seconde guerre mondiale et soulignant du même pied l’importance capitale qu’a revêtu la sécurité de l’information lors de cette époque. \\ 
Puis, la seconde guerre mondiale laissa place à la guerre froide, une guerre d’opinion opposant deux blocs : d’un côté les états unis démocrates et de l’autre les russes, communistes. Ce fut la naissance d’une course à l’armement et à l’avancée technologique pour dominer le camp adverse. 
C’est durant cette période que les premiers ordinateurs sont créés. A cette époque, ils sont beaucoup plus utilisés comme outils de calcul pour des applications scientifiques et n’étaient pas très répandus. C’était des systèmes mono-utilisateurs logés dans de grande salle et il n’y avait pas communication entre ces différentes machines. La sécurité n’était pas une priorité et n’impliquait que le fait de sécuriser les salles où étaient installées ces machines. 
Du fait de leur puissance et de leurs nombreux avantages, de plus en plus d’ordinateurs et de systèmes d’exploitation furent créés. De même, la cryptographie entre dans une nouvelle ère: des techniques de chiffrement plus avancées sont mises en place grâce à la puissance de calcul des ordinateurs. C’est la naissance de la cryptographie moderne [1].
Avec la prolifération des terminaux distants sur les ordinateurs commerciaux, le contrôle physique de l'accès à la salle informatique n'était plus suffisant. En réponse à cela, des systèmes de contrôle d'accès logique ont été développés. ( Un système de contrôle d'accès maintient une table en ligne des utilisateurs autorisés. Un enregistrement d'utilisateur type stocke le nom de l'utilisateur, son numéro de téléphone, son numéro d'employé et des informations sur les données auxquelles l'utilisateur était autorisé à accéder et les programmes qu'il était autorisé à exécuter. Un utilisateur peut être autorisé à afficher, ajouter, modifier et supprimer des enregistrements de données dans différentes combinaisons pour différents programmes.) Dans le même temps, les gestionnaires de système ont reconnu l'importance de pouvoir se remettre de catastrophes pouvant détruire le matériel et les données. Les centres de données ont commencé à faire régulièrement des copies sur bande de fichiers pour le stockage hors site. Les gestionnaires de centre de données ont également commencé à élaborer et à mettre en œuvre des plans de reprise après sinistre. Ce sont les premières politiques de sécurité entreprises. Cependant, même avec un tel système en place, de nouvelles vulnérabilités ont été reconnues au cours des années suivantes. Il fallait des systèmes plus fiables. 
Multics, un système d’exploitation multi utilisateurs fut créé en 1964. Ce fut la première fois que la problématique de la sécurité de l’information fut prise en compte en amont. En effet, dès la conception de Multics, les décisions prises (langages de programmation, architecture du noyau, etc.) prenaient en compte les exigences de sécurité. Les fonctions de sécurité de Multics comprenaient également le chiffrement des mots de passe, des audits de connexion et des procédures de maintenance logicielle. Les mots de passe dans Multics n’étaient jamais stockés en texte clair. Lorsqu'un utilisateur entrait son mot de passe, ce mot de passe était chiffré, puis comparé au mot de passe stocké sur le système. Cela permit de garder les mots de passes des utilisateurs en cas de dump système. De même, un journal d'audit de connexion enregistrait l'heure, la date et le terminal de chaque tentative de connexion, et notifiait à l'utilisateur le nombre de tentatives de mot de passe incorrectes sur son compte depuis la dernière connexion réussie. Enfin, des procédures de maintenance logicielle, telles que la vérification du nouveau logiciel permettaient de maintenir le système sûr et épargné des régressions de sécurité.
Au début des années 1970, alors que l’armée américaine était à la recherche de systèmes informatiques multi utilisateurs capables de protéger les informations classifiées, Multics lui fut recommandé. 
A cette époque, pour éprouver la sécurité des systèmes en place, il était très courant de faire appel à des Tiger Team. Il s’agissait d’experts rassemblés pour gérer des situations spéciales, régler des problèmes spécifiques le plus rapidement possible. Au début, leur travail consistait surtout en des revues manuelles de code pour détecter la source des bugs. Un peu plus tard, ils ont commencé à utiliser le «  pentest » ou test d’intrusion.( C’est une méthode permettant d’évaluer la sécurité d’un système informatique à travers des tentatives d’intrusion à la manière d’un attaquant.)
Dès 1969, l’ARPA (Advanced Research Project Agency), une agence dédiée aux projets de recherche avancée renommée plus tard en DARPA (Defense Advanced Research Project Agency) arriva à interconnecter les ordinateurs de quatre universités en un réseau afin de leur permettre de partager leurs résultats de recherche : ce réseau fut nommé l’Arpanet. Dans Arpanet, les utilisateurs se connaissaient plus ou moins et étaient pour la majorité des académiciens : la sécurité n’était pas un problème majeur dans ce « réseau d’amis ». C’est ce simple réseau de quatre nœuds sans aucune préoccupation de sécurité qui conduisit plus tard à la naissance d’Internet et du World Wide Web. 
Vers la fin des années 1970, l’analyse de codes source, qui était faite manuellement, vit une révolution. Lint, le premier outil d’analyse de codes source automatisée apparut. Initialement, il était destiné aux codes sources écrits en langage C. Lint était pratique pour trouver des bugs potentiels, mais était très lent et n'était pas équipé de la vue complète du programme. Il ne pouvait analyser qu'un seul fichier à la fois. Lint a ouvert la voie à la première génération d’outils destinés à la sécurité des applications informatiques qui, bien qu'ils aient été utiles pour trouver des bugs spécifiques, étaient assez maladroits et ne faisaient pas mieux que l'analyse manuelle.
La décennie 1970 vit également l’apparition des premiers micro-ordinateurs. Au début, parce qu’ils étaient entièrement autonomes et généralement sous le contrôle d'un seul individu, il y avait peu de problèmes de sécurité. Très rapidement ils passent d'un passe-temps pour les passionnés d'informatique en un sérieux outil de travail. A partir de ce moment, des logiciels commencent à être créés pour les ordinateurs. L’on sait que pour cela, il fallait écrire du code source parfois enclin à des bugs et à des vulnérabilités de sécurité.


\section{Evolutions récentes}




\clearpage 
%=========================================================