\chapter{Définitions}
%%\label{chap:intro}

\section{Sécurité informatique}

La sécurité peut être définie comme étant une situation, un état dans laquelle quelqu’un ou quelque chose n’est exposé à aucun danger, à aucun risque d’agression, de détérioration ou encore par le long processus visant à atteindre cet état. 
Lorsqu’on parle de sécurité dans le domaine des technologies de l’information et de la communication, on fait très souvent allusion à la sécurité de l’information. La sécurité de l'information, en anglais Information Security abrégé Infosec consiste en la mise en place d’un ensemble de stratégies pour gérer les processus, les outils et les politiques nécessaires pour prévenir, détecter, documenter et contrer les menaces à l'information. La sécurité de l’information recouvre donc toutes les techniques permettant d’assurer la protection de l’information.  La sécurité de l’information se fonde sur 3 principes fondamentaux :
-	la confidentialité : c’est le fait d’assurer que l’information ne puisse être accessible qu’à ceux qui ont l’autorisation de la consulter. Cela sous-entend le fait de rendre inintelligible cette information aux personnes non autorisées ;
-	l’intégrité : assurer que l’information n’est pas modifiable par un tiers non autorisé. Elle consiste à certifier que les données n’ont pas été détruites ou altérées tant de façon intentionnelle qu’accidentelle ;
-	la disponibilité : assurer que l’information est accessible en temps voulu par ceux qui en ont l’autorisation. Ne pas pouvoir accéder à une information en temps voulu est semblable à la non-possession de celle-ci.
Comme principes supplémentaire, nous notons :’
-	l’authentification : elle consiste à assurer l’identité d’un tiers et permet de garantir qu’un tiers est bien celui qu’il prétend être ;
-	la non-répudiation : le fait de ne pas pouvoir nier une action faite sur le système


\section{Cryptographie}

La sécurité informatique est un domaine pluridisciplinaire. En effet, pour arriver à ses buts, elle a, tout au cours utilisé, entre autres, la cryptographie. La cryptographie peut être définie comme un art et une science permettant de concevoir des techniques pour garder le secret des messages transmis. Voici les problèmes que doit résoudre la cryptographie :
-	la confidentialité
-	l’intégrité
-	l’authentification
On voit ainsi que la sécurité informatique et la cryptographie partagent des objectifs similaires. Et c’est pour cette raison que tout au long de l’histoire, elle a été utilisée dans le domaine de la sécurité informatique. De même, les évolutions dans le domaine de la sécurité informatique ont souvent été rendus possibles grâce aux avancées de la cryptographie.
On distingue :
-	la cryptographie classique qui décrit la période d’avant les ordinateurs. Elle traite des systèmes reposant sur les lettres et les caractères d’une langue naturelle. Dans cette famille, on retrouve le chiffrement par substitution qui consiste, à remplacer, sans en bouleverser l’ordre les symboles d’un texte clair par d’autres symboles [1] et le chiffrement par transposition qui repose sur le bouleversement de l’ordre des symboles du message clair. Les techniques de chiffrement les plus connues dans cette famille sont le chiffrement de César et le chiffrement de Vigenère.
-	 la cryptographie moderne qui utilise la puissance de calcul des ordinateurs pour affiner ces techniques de chiffrement. Dans cette famille, nous avons le chiffrement symétrique qui utilise une même clé pour le chiffrement et le déchiffrement (DES en est la technique la plus connue) et le chiffrement asymétrique qui utilise des clés différentes pour le chiffrement et le déchiffrement. RSA est l’algorithme de chiffrement asymétrique le plus utilisé


\clearpage 
%=========================================================