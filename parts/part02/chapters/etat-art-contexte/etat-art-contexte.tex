\section{Contexte}
Compte tenu de leur rôle essentiel dans nos sociétés et nos économies modernes, les ordinateurs, les téléphones mobiles et l'internet doivent fonctionner ensemble correctement tout en fournissant un cadre qui protègent tout un chacun. \\
Il faut donc adopter des mesures drastiques afin d’assurer la sécurité lors de nos interactions avec ces différents outils qui font partie aujourd’hui de notre quotidien. Ces mesures sont prises sur deux aspects : \\
- l’aspect juridique : de nouvelles lois sont adoptées; \\
- l’aspect technique : des services de protection des données sont créés aux échelles nationale, sous régionale et internationale.
\subsection{Contexte juridique}
Dans le cadre juridique, on parle le plus souvent de données à caractère personnel ou données personnelles. La notion de données à caractère personnel est définie comme étant toute information relative à une personne physique identifiée, génétique, psychique ou identifiable directement ou indirectement, par référence à un numéro d’identification ou à un ou plusieurs éléments propres à son identité physique, physiologique, génétique, psychique, culturelle, sociale ou économique.
Les premières législations sur la protection des données personnelles ont été adoptées en Allemagne (1971), en Suède (1973), en France (1978), au Luxembourg (1979) et au Canada (1979).\\
En Afrique, la protection des données à caractère personnel est promue par plusieurs organismes communautaires même si les lois sur la protection des données sont relativement récentes. Le tableau \ref{table:5.1} décrit l’ordre d’adoption des lois par différents pays. \cite{pdp}\\
\begin{table}[hbt!]
	\centering
	\begin{tabular}{| l | c |} 
		\hline
		Pays & Année \\ [0.5ex] 
		\hline
		Seychelles & 1988 \\ 
		\hline
		Cap-Vert & 2001 \\
		\hline
		Zimbabwe & 2002 \\
		\hline
		Burkina Faso ; Tunisie ; Iles Maurice & 2004 \\
		\hline
		Sénégal & 2008 \\
		\hline
		Bénin ; Maroc & 2009 \\
		\hline
		Ghana & 2010 \\
		\hline
		Gabon ; Angola & 2011 \\
		\hline
		Mali ; Côte d’Ivoire ; Afrique du Sud ; Lesotho & 2013 \\
		\hline
		Madagascar ; Comores & 2014 \\
		\hline
		Tchad & 2015 \\
		\hline
		Guinée Equatoriale & 2016 \\
		\hline
		Niger & 2017 \\ [1ex] 
		\hline
	\end{tabular}
	\caption{Années d'adoption de lois sur les données personnelles dans différents pays en Afrique}
	\label{table:5.1}
\end{table}
\\En Afrique de l’Ouest, la CEDEAO (Communauté Économique des États de l'Afrique de l'Ouest) et l’UEMOA (Union Economique et Monétaire Ouest Africaine) \nomenclature{CEDEAO}{Communauté Économique des États de l'Afrique de l'Ouest} \nomenclature{UEMOA}{Union Economique et Monétaire Ouest Africaine} sont intervenus en tant qu’acteur régional dans la réglementation des données personnelles en adoptant des mesures juridiques tout comme l’Union Africaine au niveau continental. \\
Au Sénégal, nous avons la loi n° 2008-11 du 25 Janvier 2008 portant loi d’orientation relative à la société de l’information qui  définit un cadre général pour adapter le droit sénégalais aux besoins de la société de l’information. \\
Nous avons aussi la loi n° 2008-11 du 25 Janvier 2008 portant sur la cybercriminalité .La cybercriminalité ou criminalité informatique concerne toute infraction qui implique l’utilisation des technologies de l'information et de la communication. \\
Il y a aussi la loi sur les transactions électroniques qui vise, de façon globale, à favoriser le développement du commerce par les Technologies de l’Information et de la Communication en posant des règles précises. \\
Il y a surtout la loi n° 2008-12 du 25 Janvier 2008 sur la protection des données à caractère personnel qui est le principal corpus protecteur des dites données.\\
Ce cadre juridique définit des exigences de sécurité que doivent respecter les responsables des traitements des données à caractère personnel. Elles imposent à ces derniers des obligations de confidentialité, de sécurité, de conservation et de pérennité des données. En effet, tout responsable de traitement de données personnelles se doit de mettre en œuvre les mesures techniques et organisationnelles adéquates pour protéger les données collectées contre la destruction accidentelle ou illicite, la perte, l’altération, la diffusion ou l’accès non autorisé notamment lorsque le traitement comporte des transmissions des données dans un réseau, ce qui est presque toujours le cas, ainsi que contre toute autre forme de traitement illicite. Ces mesures doivent assurer un niveau de sécurité approprié au regard des risques présentés par le traitement et la nature des données manipulées, empêchant le tiers de procéder à leur modification, à leur altération ou à leur consultation sans autorisation.\\
Cette obligation se traduit donc par la nécessité de mettre en œuvre des mesures de sécurité physique (verrous des salles serveur, coffres forts, etc.) et des mesures de sécurité logique (contrôles d’accès, cryptage des données, etc.).\\
Au Sénégal, le fait de procéder à des traitements automatisés de données personnelles sans prendre toutes les précautions utiles pour préserver leur sécurité est passible d’une peine d’emprisonnement de 1 à 7 ans et d’une amende allant de 500 000 à 10 000 000 de francs CFA. Pour assurer le respect des règles juridiques quant à la protection des données personnelles, il est institué par les différentes lois, un organisme responsable. Au Sénégal, c’est la CDP (Commission des Données Personnelles) \nomenclature{CDP}{Commission des Données Personnelles} qui joue ce rôle.

\subsection{Contexte technique et organisationnel}
\subsubsection{Organismes nationaux}
De partout dans le monde, des CERT sont mis en place pour prendre en compte les incidents susceptibles de se produire sur Internet. Un CERT est un Centre d’alerte et de réaction aux attaques informatiques ciblant les entreprises ou administrations mais dont les informations sont généralement accessibles à tous. Le premier CERT (CERT-CC) a été créé aux Etats-Unis à l’Université Carnegie-Mellon en réponse à l’attaque du ver Morris. \\
Le but des CERT est de répondre aux incidents de sécurité informatique, de coordonner la communication entre experts lors de ces incidents de sécurité, de signaler les vulnérabilités et promouvoir des pratiques de sécurité efficaces dans toute la communauté internet afin de prévenir les incidents futurs. \\
Dans presque tous les pays développés, nous avons au moins un CERT national :
\begin{itemize}
	\itemtirait CERT-FR en France ;
	\itemtirait CERTBund en Allemagne ; 
	\itemtirait JPCERT au Japon ;
	\itemtirait UKCERT au Royaume-Uni.
\end{itemize}
Cette liste est loin d'être exhaustive.
En Afrique, des efforts ont été récemment faits  dans ce domaine même s’il faudrait que plus de pays africains mettent en place des CERT. Nous avons notamment :
\begin{itemize}
	\itemtirait le CSIRT du Kenya ;
	\itemtirait le CERT des îles Maurice ;
	\itemtirait l’ECS-CSIRT de l’Afrique du Sud ;
	\itemtirait le TunCERT de la Tunisie ;
	\itemtirait le CI-CERT de la Cote d’Ivoire ;
	\itemtirait le CERT-GH du Ghana ;
	\itemtirait le DZ-CERT de l’Algérie ;
	\itemtirait ou encore le maCERT du Maroc.
\end{itemize}
Nous avons aussi le Forum africain des équipes de réponse aux incidents informatiques (AfricaCERT) qui veut asseoir les bases d’une coopération dynamique entre les équipes nationales africaines de CERT pour lutter contre le phénomène de la cybercriminalité qui menace l’économie, l’image et la jeunesse africaines.\\
Au Sénégal, il urge de mettre en place un CERT. A ce sujet, un accord a été signé entre Suricate, une entreprise luxembourgeoise spécialisée dans la cyber sécurité et l'ADIE (Agence De l'Informatique de l'Etat) pour la mise en place d'un CERT national. \nomenclature{ADIE}{Agence De l'Informatique de l'Etat}\\
Les CERTs à travers le monde sont des entités indépendantes, bien qu'il puisse y avoir des activités coordonnées entre les groupes. Ces dernières années, de nombreux CERT ont vu le jour et font partie du Forum des équipes de réaction aux incidents de sécurité informatique (FIRST \nomenclature{FIRST}{Forum of Incident Response and Security Teams} - Forum of Incident Response and Security Teams). \\
FIRST est une organisation de premier plan et un leader mondial reconnu dans la réponse aux incidents de sécurité informatique. L'appartenance à FIRST permet aux CERTs d'intervenir plus efficacement face aux incidents de sécurité en fournissant un accès aux meilleures pratiques de sécurité, à des outils de gestion de la sécurité et à une communication de confiance entre les équipes membres. Il s'agit d'une confédération internationale de CERTs de confiance qui gèrent de manière coopérative les incidents de sécurité informatique et favorisent les programmes de prévention des incidents. Ils travaillent tous pour un objectif commun de sécurité informatique. En outre, de nombreuses sociétés privées de d’édition de logiciels anti-virus ont des divisions qui jouent le rôle de CERT.\\
Il existe aussi en plus des CERTS, des autorités techniques nationales, sous-régionales et régionales chargées de préserver la sécurité de l’information aux niveaux nationales, sous-régionales et régionales. L’enjeu de ces autorités nationales est de préserver la souveraineté et l’autonomie de décision et d’action dans le domaine informatique et protéger l’ensemble des infrastructures critiques. \\
En parallèle à ces initiatives étatiques, il existe des organisations indépendantes qui œuvrent pour une meilleure sécurité de l’information. Parmi celles-ci, les plus connues sont Mitre, Sans Institute et OWASP.
\subsubsection{Organismes indépendants}
\textbf{\RIGHTarrow Mitre Corporation}\\
Mitre Corporation est une organisation à but non lucratif travaillant dans l’intérêt public fondé en 1958. Elle gère les centres de recherche et de développement financés par l’état fédéral notamment celui du Département de la Défense chargé de la sécurité nationale aux Etats Unis. Les FFRDCs (Federally Funded Research and Development Center) \nomenclature{FFRDC}{Federally Funded Research and Development Center} fournissent des services dans les domaines de l'acquisition et de l'analyse de systèmes notamment sur la cyber-sécurité et la mise en réseau mondiale. Ils s'engagent également dans la recherche et le développement de technologies telles que la biosécurité et l'informatique quantique.\\
Mitre Corporation entretient une Cyber-académie avec des cours en ligne. Elle maintient aussi la liste des CVE\footnote{La liste CVE est une liste d’entrées, chacune contenant un numéro d’identification, une description et au moins une référence publique pour les vulnérabilités de cybersécurité connues du public.} (Central Vulnerabilities and Exposures) \nomenclature{CVE}{Central Vulnerabilities and Exposures} avec le sponsoring du Département de la Sécurité Intérieure, la liste CCE\footnote{La liste CCE est une liste contenant des identifiants uniques pour les problèmes de configuration système liés à la sécurité.} (Common Configuration Enumeration) \nomenclature{CCE}{Common Configuration Enumeration}, la liste CAPEC\footnote{La liste CAPEC est un dictionnaire complet et une taxonomie de classification des attaques connues.} (Common Attack Pattern Enumeration and Classification) \nomenclature{CAPEC}{Common Attack Pattern Enumeration and Classification} et la liste CWE\footnote{La liste CWE est une liste recensant les failles logicielles connues} (Common Weakness Enumeration) \nomenclature{CWE}{Common Weakness Enumeration}. Mitre Corporation est un partenaire très proche des États-Unis.\\

\textbf{\RIGHTarrow Sans Institute}\\
Sans Institute est une organisation privée à but lucratif qui offre des formations et des certifications en sécurité de l'information et en cyber sécurité à travers le monde fondée en 1989. Elle maintient le plus grand référentiel d'informations sur la sécurité dans le monde et est également le plus grand organisme de certification. Sans Institute fournit une vaste collection de documents de recherche sur la sécurité et supervise un système d’alerte d’attaques : Internet Storm Center. Son programme GIAC (Global Information Assurance Certification) \nomenclature{GIAC}{Global Information Assurance Certification} fournit un moyen normalisé de garantir les connaissances et les compétences d'un professionnel de la sécurité. La majorité des ressources de Sans Institute sont libres mais pas toutes.\\

\textbf{\RIGHTarrow PCI Standard Council}\\
Le PCI Standard Council (Payment Card Industry Standard Council) \nomenclature{PCI}{Payment Card Industry} est une organisation fondée en 2006 par American Express, Discover, JCB International, MasterCard et Visa Inc. Elle recommande, maintient et évolue des normes pour la sécurité des données des titulaires de cartes dans l'industrie des cartes de paiement à travers le monde. Les normes PCI Data Security aident à protéger la sécurité des données de carte de paiement bancaires. Ils définissent les exigences opérationnelles et techniques pour les organisations acceptant ou traitant des transactions de paiement, et pour les développeurs de logiciels et les fabricants d'applications et de dispositifs utilisés dans ces transactions.\\

\textbf{\RIGHTarrow Web Application Security Consortium}\\
Le WASC (Web Application Security Consortium) \nomenclature{WASC}{Web Application Security Consortium} est une organisation mondiale consacrée à l'établissement, au perfectionnement et à la promotion des normes de sécurité sur Internet. Le consortium, créé en janvier 2004, est composé de membres indépendants ainsi que de membres associés à des entreprises, des organismes gouvernementaux et des établissements universitaires.\\
Le champ d'actions du WASC comprend la recherche et la publication d'informations sur les problèmes de sécurité des applications Web. L’organisation informe les particuliers et les entreprises sur ces problèmes et sur les mesures à prendre pour lutter contre des menaces spécifiques. Elle accompagne également les utilisateurs d’Internet et les organisations dévouées à la sécurité des applications Web. Le WASC est une organisation indépendante, bien que les membres puissent appartenir à des sociétés impliquées dans la recherche, le développement, la conception et la distribution de produits liés à la sécurité Web.\\

\textbf{\RIGHTarrow OWASP}\\
OWASP est une organisation indépendante à but non lucratif créé en 2001 par un certain Mark Curphey. Il s'agit d'une communauté en ligne ouverte et libre travaillant sur la sécurité des applications Web.\\
OWASP est aujourd'hui reconnue dans le monde de la sécurité des systèmes d'information pour ses travaux et recommandations liées aux applications Web. Ces recommandations vont dans le sens de bonnes/mauvaises pratiques de développement, d’une base sérieuse en termes de statistiques, et d’un ensemble de ressources amenant à une base de réflexion sur la sécurité.\\

\textbf{\RIGHTarrow Choix d'Owasp}\\
Comme nous l'avons présenté, il existe plusieurs organismes indépendants œuvrant pour une meilleure sécurité des applications web et mobile. + comparaison + tableau \\
Evaluer la sécurité d’une application Web en se basant sur le Top 10 de l’OWASP est une pratiquement largement acceptée. De nombreuses organisations, notamment le PCI Standard Council, l'Institut national des normes et technologies (NIST - National Institute of Standards and Technology) \nomenclature{NIST}{National Institute of Standards and Technology} et la Commission Fédérale du Commerce (FTC - Federal Trade Commission) \nomenclature{FTC}{Federal Trade Commission}, citent régulièrement le Top 10 d'OWASP comme un guide de référence intégral pour atténuer les risques de sécurité des applications web et mobiles et respecter les principales normes de sécurité.