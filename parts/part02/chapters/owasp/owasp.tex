\section{Owasp}
\subsection{Présentation}
OWASP (Open Web Application Security Project) est une communauté en ligne ouverte et libre travaillant sur la sécurité des applications Web. OWASP se propose de permettre aux organisations de concevoir, développer, acquérir, exploiter et maintenir des applications logicielles fiables.\\
OWASP est aujourd'hui reconnue dans le monde de la sécurité des systèmes d'information pour ses travaux et recommandations liées aux applications Web. Ces recommandations vont dans le sens de bonnes/mauvaises pratiques de développement, d’une base sérieuse en termes de statistiques, et d’un ensemble de ressources amenant à une base de réflexion sur la sécurité. Des outils sont aussi proposés pour effectuer des audits de sécurité.
La Fondation OWASP, un organisme de bienfaisance à but non lucratif soutient les efforts de l'OWASP à travers le monde. Tous les outils, documents, forums et chapitres d'OWASP sont gratuits et ouverts à toute personne intéressée par l'amélioration de la sécurité des applications.\\
OWASP est libre de toute pression commerciale et n'est affilié à aucune entreprise de technologie. Cela lui permet de fournir des informations objectives, pratiques et effectives sur la sécurité des applications. 
Les professionnels de la sécurité peuvent intégrer les recommandations d'OWASP dans leurs travaux. Les fournisseurs de service sécurité peuvent baser leurs produits et services sur les standards OWASP. Les consommateurs peuvent utiliser les normes comme documents de référence pour tester les applications ou les services qu'ils utilisent.
\subsection{Origines}
OWASP a été créé par un certain Mark Curphey le 9 septembre 2001. Son but initial étatit de lancer un projet pour définir une méthodologie de test standard pour la sécurité des applications Web. Il continue de définir des recommandations de sécurité, des spécifications et des explications dans des domaines clés de la sécurité des applications Web.
Sa philosophie est d'être à la fois libre et ouverte à tous. Elle a pour vocation de publier des recommandations de sécurisation Web et de proposer aux internautes, administrateurs et entreprises des méthodes et outils de référence permettant de contrôler le niveau de sécurisation de ses applications Web.
\subsection{Contexte}
De nos jours, le développement de produits informatiques est fermement axé sur la vitesse. La course du Time-To-Market est extrêmement compétitive. Pour innover, les entreprises développent à un rythme effréné, en établissant des méthodologies qui permettent de perfectionner leur logiciel tout en réduisant le temps de développement. La sécurité, cependant, est souvent une réflexion secondaire pour les développeurs et les clients poussent toujours à livrer plus rapidement. Cela ouvre la porte à des failles de sécurité, de plus en plus présentes dans les logiciels. Et le Web n’échappe pas à cet état de fait.\\ 
De même, il est souvent difficile de trouver des conseils impartiaux, objectifs et des informations pratiques aidant à la prise en compte de la sécurité dans le développement. Le marché concurrentiel de la technologie et des services a beaucoup à dire sur ce point, mais une grande partie les conseils et recommandations sont donnés pour vous orienter vers un outil ou un fournisseur de services particulier.\\
L'OWASP a été créé pour lutter contre ce problème, en offrant des conseils impartiaux et objectifs sur les meilleures pratiques et en encourageant la création de normes ouvertes.
\subsection{Projets OWASP}
Tous les projets OWASP d'outils, de documents et de bibliothèques de codes sont organisés dans les catégories suivantes : 
\begin{itemize}
	\itemcheck Projets phares : \\
	La désignation « OWASP Flagship » est attribuée aux projets qui ont démontré leur valeur stratégique pour l’OWASP et la sécurité des applications dans son ensemble.
	\itemcheck Projets de laboratoire : \\
	Les projets OWASP Labs représentent des projets qui ont produit un livrable de valeur révisé par l’OWASP.
	\itemcheck Projets d'incubation : \\
	Les projets OWASP Incubators représentent les projets qui sont encore en cours d'élaboration, avec des idées et dont le développement sont toujours en cours.
\end{itemize}
Les projets OWASP couvrent de nombreux aspects de la sécurité des applications. Elles concernent des documents, des outils, des environnements d'enseignement, des lignes directrices, des listes de vérification et d'autres documents pour aider les organisations à améliorer leur capacité à produire du code sécurisé. Les projets sont l'une des principales méthodes par lesquelles OWASP s'efforce de réaliser sa mission, qui est de rendre la sécurité plus « visible ».\\
Les projets OWASP sont animés par des bénévoles et sont ouverts à tous. Cela signifie que n'importe qui peut diriger un projet, que n'importe qui peut contribuer à un projet et que n'importe qui peut utiliser un projet. \\
Voici une liste (non exhaustive) de projets OWASP populaires, ainsi qu’une description succincte de chacun d'eux :
\begin{itemize}
	\itemcheck Owasp Testing Guide : \\
	Il s'agit d'un document de plusieurs centaines de pages destiné à aider une personne à évaluer le niveau de sécurité d'une application Web.
	\itemcheck Owasp code Review Guide : \\
	Il s'agit d'un document de plusieurs centaines de pages présentant une méthode de revue de code sécurité.
	\itemcheck Owasp Application Security Verification Standard : \\
	Le projet ASVS vise à créer un ensemble de normes commerciales permettant d'effectuer une vérification de sécurité rigoureuse d’une application au niveau applicatif.
	\itemcheck Top 10 Owasp : \\
	Il s'agit d'une liste des dix risques de sécurité les plus critiques pour les applications Web.
	\itemcheck Owasp Enterprise Security API (ESAPI) : \\ 
	\nomenclature{ESAPI}{Enterprise Security API}
	\nomenclature{API}{Application Programming Interface}
	Owasp ESAPI est une bibliothèque de contrôle de sécurité pour applications Web gratuite et opensource qui permet aux programmeurs d'écrire plus facilement des applications à faible risque en mettant à leur disposition un ensemble de fonctions de sécurité.
\end{itemize}
\subsection{Top 10 Owasp}
\subsubsection{Présentation}
Il s’agit d’un document de sensibilisation à la sécurité des applications Web. La liste résulte d’un consensus entre les experts en sécurité leaders dans le domaine, concernant les dix risques de sécurité les plus critiques pour les applications Web. Le classement de ces risques de sécurité est basé sur leur fréquence, la gravité des risques de sécurité et l'ampleur de leur impact commercial potentiel. Le Top 10 de l’OWASP a pour but d’informer sur l’existence de ces risques et de fournir des guides simplifiés sur les bonnes pratiques pour s’en prémunir. L’OWASP maintient le Top 10 depuis 2003. Il a été créé à l'origine pour aider les organisations à établir une base, un point de départ leur permettant de déterminer si leur infrastructure de sécurité est prête à résister aux principales menaces. La liste continue de servir de liste de contrôle et de standard de développement d'applications Web pour plusieurs des plus grandes organisations du monde. \\
La liste est mise à jour tous les trois ou quatre ans pour suivre le rythme des changements qui se produisent sur le marché de la sécurité des applications Web. La version la plus récente a été publiée en 2017. Celle-ci, contrairement à celles précédentes, n’est plus uniquement basé sur la « vision » de l’OWASP sur le sujet. Le processus méthodologique a été entièrement revu. Il repose ainsi sur les remontées de 500 utilisateurs et de 40 sociétés spécialisées dans le domaine de la sécurité des applications. La liste des contributeurs et les données techniques issues de leurs remontées sont disponibles en Open Source sur Github. En outre, les statistiques concernent un panel de plus de 100 000 applications et services Web.\\
Cet ensemble de risques de sécurité des applications Web largement accepté est complété par un ensemble de directives de codage et de test sécurisés. Ces guides sont disponibles sur le site de l'OWASP et s'adressent aux développeurs, architectes, chefs de projets, managers ...
\subsubsection{Démarches Concurrentes}
Le Top 10 OWASP n’est pas la seule liste de risques de sécurité existante en matière de sécurité, il y a aussi, parmi les plus populaires : 
\begin{itemize}
	\itemcheck la liste CWE : \\
	Le Common Weakness Enumeration (CWE), maintenu par Mitre Corporation, est une liste de vulnérabilités que l’on retrouve lors du développement d’applications. C'est un projet  géré par MITRE. Pour chaque entrée, le CWE fournit une description de la vulnérabilité ainsi que les étapes pour y remédier.\\
	Cependant, contrairement aux Top 10 d’OWASP qui recense les 10 risques de sécurité les plus critiques, le CWE se veut être une démarche plus globale. Au moment ou nous écrivons ces lignes, 714 vulnérabilités sont recensées sur la liste CWE. Elle peut constituer une suite à la gestion de la sécurité pour une organisation après que celle-ci ait pris en compte le Top 10.
	\itemcheck le CWE/Sans Top 25 : \\
	MITRE s'est associé à Sans Institute pour développer le CWE/Sans Top 25, une liste des 25 vulnérabilités logicielles les plus critiques. Bien que le Top 10 OWASP et le CWE/Sans Top 25 soient différents, ils partagent la plupart des mêmes vulnérabilités. En effet, là où le Top 10 adresse les risques de sécurité en faisant une approche groupée, le CWE/Sans Top 25 utilise une approche plus granulaire. Par exemple, la correspondance OWASP Top 10 - CWE/Sans Top 25 peut être faite sur le point A1 : Injection comme suit \cite{cross}:
	\begin{table}[hbt!]
		\centering
		\begin{tabular}{| c | l |} 
			\hline
			OWASP Top 10 & CWE/Sans Top 25 \\
			\hline
			\multirow{6}{4em}{A1:Injection} & CWE-78: Improper Neutralization of Special Elements Used in an OS Command \\ 
			& CWE-89: SQL Injection\\ 
			& CWE-94: Code Injection\\ 
			& CWE-434: Unrestricted Upload of File with Dangerous Type\\
			& CWE-494: Download of Code Without Integrity Check\\
			& CWE-829: Inclusion of Functionality from Untrusted Control Sphere\\
			\hline
		\end{tabular}
		\caption{Correspondance Top 10 OWASP A1 - CWE/Sans Top 25}
	\end{table}\\
	Cette correspondance peut être faite pour toutes les entrées du Top 10 d’OWASP.
\end{itemize}
\subsection{OWASP ESAPI}
\subsubsection{Présentation}
OWASP ESAPI est une bibliothèque de contrôle de sécurité pour applications Web gratuite et opensource. La bibliothèque OWASP ESAPI a été conçue pour aider les programmeurs à intégrer plus facilement la sécurité dans des applications existantes  en mettant à leur disposition un ensemble de fonctions de sécurité. La bibliothèque ESAPI constituent également une base solide pour les nouveaux développements.\\
Le code source de la bibliothèque OWASP ESAPI est sous licence BSD La documentation du projet est sous licence Creative Commons. L'on peut utiliser ou modifier OWASP ESAPI comme bon nous semble et même l'inclure dans des produits commerciaux.
\subsubsection{Architecture}
La bibliothèque OWASP ESAPI est conçue de la sorte suivante:
\begin{itemize}
	\itemcheck un ensemble d'interfaces de contrôle de sécurité. Ces interfaces ne contiennent pas de logique d'application. Ils définissent par exemple des types de paramètres qui sont passés aux types de contrôles de sécurité. Il n'y a pas de propriétés informationnelles ou de logique contenues dans ces interfaces.
	\itemcheck une implémentation de référence pour chaque contrôle de sécurité c'est-à-dire un ensemble de classes qui implémentent les interfaces prédéfinies et qui contiennent une certaine logique applicative. Cependant, ces implémentations ne sont ni orientées application, ni orientées organisation.
	\itemcheck éventuellement une implémentation propre à l'organisation ou aux applications pour chaque contrôle de sécurité.
\end{itemize}
\subsubsection{Qualité}
Il est d'une importance capitale que les contrôles de sécurité des applications soient bien en place. Une simple erreur pourrait exposer une application à de gros risques de sécurité. L'assurance par rapport à ces contrôles de sécurité provient de preuves telles que la documentation de conception, la révision de code, les tests de sécurité et autres analyses.\\
Le projet ESAPI implique une équipe d'experts de classe mondiale en sécurité logicielle issus de l'industrie de la sécurité. L'implémentation de référence est assez courte et bien structurée, environ 5 000 lignes de code bien documenté et revu en détail. Le code a été analysé dans tous les principaux outils d'analyse statique, y compris FindBugs, PMD, Ounce et Fortify et est «clean». Le projet comprend également environ 600 cas de test qui testent toutes les fonctions de sécurité.\\
Beaucoup d'organisations, et, pas des moindres, commencent à adopter la bibliothèque Owasp ESAPI pour sécuriser leurs applications. Parmi celles-ci, nous avons : American Express, Apache Foundation, Booz Allen Hamilton, Aspect Security, Coraid, The Hartford, Infinite Campus, Lockheed Martin, MITRE, US Navy. - SPAWAR, Banque mondiale, SANS Institute.
%\clearpage 
%=========================================================